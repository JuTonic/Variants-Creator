\documentclass{article}

\usepackage{geometry}
\geometry{
    a4paper,
    includehead=true,
    headsep=3.5mm,
    top=10mm,
    left=20mm,
    right=20mm,
    bottom=20mm
}

\usepackage[russian]{babel}
\sloppy
    
\usepackage{fancyhdr}
\fancyhf{}
\fancypagestyle{fancy}{
	\fancyhead[L]{\textit{Самостоятельная работа 4}}
	\fancyhead[R]{\textit{Ответы}}
	\fancyfoot[L]{\thepage}
	\fancyfoot[R]{\textit{курс "основы статистических наблюдений", 2023}}
	\renewcommand{\headrulewidth}{0.20ex}
}
\pagestyle{fancy}
        
\usepackage{fontspec-xetex}
\setmainfont{OpenSans}[
  Path    = D:/local/LaTeX/fonts/ ,
  Extension  = .ttf ,
  UprightFont  = *-Regular ,
  BoldFont  = *-Bold ,
  ItalicFont  = *-Italic ,
  BoldItalicFont = *-BoldItalic ]
    
\usepackage{booktabs}
    
\usepackage{amsmath}
\usepackage{unicode-math}
\setmathfont[Scale=1.2]{Cambria Math}
    
\usepackage{caption}
\captionsetup[table]{name=Таблица, aboveskip=3pt, labelfont={it}, font={it}}
\captionsetup[figure]{labelformat=empty, margin=0pt, skip = -12pt}

\setlength\parindent{0pt}

\usepackage{booktabs}
\setlength\heavyrulewidth{0.20ex}
\setlength\cmidrulewidth{0.10ex}
\setlength\lightrulewidth{0.10ex}

\usepackage{tabularx}
\newcolumntype{Y}{>{\centering\arraybackslash}X}
\def\tabularxcolumn#1{m{#1}}

\usepackage{svg}

\usepackage{tcolorbox}

\renewcommand{\baselinestretch}{1.5}

\usepackage{makecell}

\usepackage{underscore}

\begin{document}


\mbox{}

\vspace{-36pt}

\begin{center}
	\begin{tcolorbox}[colback=white, boxrule=0.20ex, sharp corners = all, height=25pt, colframe=black, valign=top]
		\begin{center}
			Фамилия Имя:\hspace{1.5pt}\rule{190pt}{0pt}\hspace{50pt}Группа:\hspace{1.5pt}\rule{60pt}{0pt}
		\end{center}
	\end{tcolorbox}
\end{center}
\vspace{3pt}

\textbf{Вариант 1}
\begin{enumerate}
\item $y = 52.9+2.96\cdot x$. При увеличении $x$ на единицу, $y$ изменится на 2.96
\item Коэффициент корреляции говорит о наличии или отсутствии лишь \textit{линейной} взаимосвязи. $x_2 = a * x_1 ^ 2$  
\item $S = 11.83$
\item $\rho \approx 0.8$, $\rho_{exact} = 0.7994$. Положительная сильная линейная связь
\item $\varepsilon = 2.4$. При увеличении цены маркера на 1\%, срок его службы вырастет на 2.4\%
\end{enumerate}

\textbf{Вариант 2}
\begin{enumerate}
\item $S = 10.2$
\item $\varepsilon = 1.5$. При увеличении $x$ на 1\%, $y$ изменится на 1.5\%
\item Положительная умеренная (средняя) линейная. $\symit{\rho_{exact} = 0.5892}$
\item $\rho \approx -0.1$, $\rho_{exact} = -0.1003$. Линейная связь отсутствует
\item $y = 74.5+-4.0\cdot x$. При увеличении максимальной частоты процессора на 1 ГГц, скорость выполнения программы изменится на -4.0
\end{enumerate}

\textbf{Вариант 3}
\begin{enumerate}
\item $\varepsilon = 0.8$. При увеличении цены маркера на 1\%, срок его службы вырастет на 0.8\%
\item $y = 70.55+-0.11\cdot x$. Каждая дополнительная выкуренная сигарета изменяет среднее количество лет, которое человек проживёт на -0.11
\item Коэффициент корреляции говорит о наличии или отсутствии лишь \textit{линейной} взаимосвязи. $x ^ 2 + y ^ 2 = r^2$
\item $S = 11.36$
\item $\rho \approx -0.32$, $\rho_{exact} = -0.3138$. Отрицательная умеренная (средняя) линейная связь
\end{enumerate}

\textbf{Вариант 4}
\begin{enumerate}
\item $S = 40.24$
\item $\rho \approx -0.8$, $\rho_{exact} = -0.7428$. Отрицательная сильная линейная связь
\item $\varepsilon = 2.1$. При увеличении $x$ на 1\%, $y$ изменится на 2.1\%
\item $y = 0.05 + -0.198\cdot x$. При увеличении ВВП страны на 1 млрд. долларов, ИЧР в среднем вырастёт на -0.198
\item Лаборант был неправ. Коэффициент корреляции говорит о наличии или отсутствии лишь \textit{линейной} взаимосвязи. $y = cos(x)$
\end{enumerate}

\textbf{Вариант 5}
\begin{enumerate}
\item $\varepsilon = 0.4$. При увеличении цены маркера на 1\%, срок его службы вырастет на 0.4\%
\item Отрицательная умеренная (средняя) линейная связь. $\symit{\rho_{exact} = -0.5162}$
\item $y = 0.052 + -0.0552\cdot x$. При увеличении ВВП страны на 1 млрд. долларов, ИЧР в среднем вырастёт на -0.0552
\item $\rho \approx -0.1$, $\rho_{exact} = -0.0995$. Линейная связь отсутствует
\item $S = 39.59$
\end{enumerate}

\textbf{Вариант 6}
\begin{enumerate}
\item $\rho \approx -0.8$, $\rho_{exact} = -0.8729$. Отрицательная сильная линейная связь
\item $\varepsilon = 3.1$. При увеличении $x$ на 1\%, $y$ изменится на 3.1\%
\item $y = 52.62+2.97\cdot x$. При увеличении $x$ на единицу, $y$ изменится на 2.97
\item Коэффициент корреляции говорит о наличии или отсутствии лишь \textit{линейной} взаимосвязи. $x ^ 2 + y ^ 2 = r^2$
\item $S = 16.58$
\end{enumerate}

\textbf{Вариант 7}
\begin{enumerate}
\item $\rho \approx 0.6$, $\rho_{exact} = 0.5993$. Положительная умеренная (средняя) линейная
\item $y = 60.0+-2.0\cdot x$. При увеличении максимальной частоты процессора на 1 ГГц, скорость выполнения программы изменится на -2.0
\item Лаборант был неправ. Коэффициент корреляции говорит о наличии или отсутствии лишь \textit{линейной} взаимосвязи. $y = cos(x)$
\item $S = 10.34$
\item $\varepsilon = 1.1$. При увеличении цены маркера на 1\%, срок его службы вырастет на 1.1\%
\end{enumerate}

\textbf{Вариант 8}
\begin{enumerate}
\item $y = 71.06+-0.09\cdot x$. Каждая дополнительная выкуренная сигарета изменяет среднее количество лет, которое человек проживёт на -0.09
\item $S = 10.2$
\item $\varepsilon = 1.2$. При увеличении $x$ на 1\%, $y$ изменится на 1.2\%
\item $\rho \approx -0.5$, $\rho_{exact} = -0.4938$. Отрицательная умеренная (средняя) линейная связь
\item Коэффициент корреляции говорит о наличии или отсутствии лишь \textit{линейной} взаимосвязи. $x_2 = a * x_1 ^ 2$  
\end{enumerate}

\textbf{Вариант 9}
\begin{enumerate}
\item $\rho \approx -0.8$, $\rho_{exact} = -0.8165$. Отрицательная сильная линейная связь
\item Положительная умеренная (средняя) линейная. $\symit{\rho_{exact} = 0.6208}$
\item $y = 55.8+-1.0\cdot x$. При увеличении максимальной частоты процессора на 1 ГГц, скорость выполнения программы изменится на -1.0
\item $\varepsilon = 1.5$. При увеличении цены маркера на 1\%, срок его службы вырастет на 1.5\%
\item $S = 35.61$
\end{enumerate}

\textbf{Вариант 10}
\begin{enumerate}
\item $\rho \approx 0.6$, $\rho_{exact} = 0.5987$. Положительная умеренная (средняя) линейная
\item $S = 7.35$
\item $y = 72.755+-0.155\cdot x$. Каждая дополнительная выкуренная сигарета изменяет среднее количество лет, которое человек проживёт на -0.155
\item $\varepsilon = 1.1$. При увеличении $x$ на 1\%, $y$ изменится на 1.1\%
\item Лаборант был неправ. Коэффициент корреляции говорит о наличии или отсутствии лишь \textit{линейной} взаимосвязи. $y = cos(x)$
\end{enumerate}

\textbf{Вариант 11}
\begin{enumerate}
\item $\rho \approx -0.4$, $\rho_{exact} = -0.3973$. Отрицательная умеренная (средняя) линейная связь
\item $\varepsilon = 0.8$. При увеличении цены маркера на 1\%, срок его службы вырастет на 0.8\%
\item $S = 9.22$
\item $y = 0.016 + -0.1053\cdot x$. При увеличении ВВП страны на 1 млрд. долларов, ИЧР в среднем вырастёт на -0.1053
\item Коэффициент корреляции говорит о наличии или отсутствии лишь \textit{линейной} взаимосвязи. $x_2 = a * x_1 ^ 2$  
\end{enumerate}

\textbf{Вариант 12}
\begin{enumerate}
\item $\rho \approx 0.2$, $\rho_{exact} = 0.1995$. Положительная слабая линейная связь
\item $\varepsilon = 2.2$. При увеличении $x$ на 1\%, $y$ изменится на 2.2\%
\item Коэффициент корреляции говорит о наличии или отсутствии лишь \textit{линейной} взаимосвязи. $x_2 = a * x_1 ^ 2$  
\item $y = 47.86+3.04\cdot x$. При увеличении $x$ на единицу, $y$ изменится на 3.04
\item $S = 6.48$
\end{enumerate}

\textbf{Вариант 13}
\begin{enumerate}
\item $y = 0.1696 + -0.0464\cdot x$. При увеличении ВВП страны на 1 млрд. долларов, ИЧР в среднем вырастёт на -0.0464
\item Коэффициент корреляции говорит о наличии или отсутствии лишь \textit{линейной} взаимосвязи. $x ^ 2 + y ^ 2 = r^2$
\item $\rho \approx -0.4$, $\rho_{exact} = -0.4364$. Отрицательная умеренная (средняя) линейная связь
\item $S = 9.75$
\item $\varepsilon = 0.6$. При увеличении цены маркера на 1\%, срок его службы вырастет на 0.6\%
\end{enumerate}

\textbf{Вариант 14}
\begin{enumerate}
\item $S = 11.22$
\item $\varepsilon = 2.6$. При увеличении $x$ на 1\%, $y$ изменится на 2.6\%
\item $y = 65.2+-2.5\cdot x$. При увеличении максимальной частоты процессора на 1 ГГц, скорость выполнения программы изменится на -2.5
\item Лаборант был неправ. Коэффициент корреляции говорит о наличии или отсутствии лишь \textit{линейной} взаимосвязи. $y = cos(x)$
\item $\rho \approx -0.8$, $\rho_{exact} = -0.7845$. Отрицательная сильная линейная связь
\end{enumerate}

\textbf{Вариант 15}
\begin{enumerate}
\item $\varepsilon = 3.1$. При увеличении $x$ на 1\%, $y$ изменится на 3.1\%
\item $y = 51.3+3.0\cdot x$. При увеличении $x$ на единицу, $y$ изменится на 3.0
\item $\rho \approx 0.75$, $\rho_{exact} = 0.7493$. Положительная сильная линейная связь
\item Линейная связь отсутствует. $\symit{\rho_{exact} = 0.0317}$
\item $S = 10.3$
\end{enumerate}

\textbf{Вариант 16}
\begin{enumerate}
\item $S = 13.23$
\item $\varepsilon = 0.9$. При увеличении цены маркера на 1\%, срок его службы вырастет на 0.9\%
\item $\rho \approx -0.32$, $\rho_{exact} = -0.3184$. Отрицательная умеренная (средняя) линейная связь
\item Коэффициент корреляции говорит о наличии или отсутствии лишь \textit{линейной} взаимосвязи. $x ^ 2 + y ^ 2 = r^2$
\item $y = 49.54+3.02\cdot x$. При увеличении $x$ на единицу, $y$ изменится на 3.02
\end{enumerate}

\textbf{Вариант 17}
\begin{enumerate}
\item $y = 69.625+-0.0625\cdot x$. Каждая дополнительная выкуренная сигарета изменяет среднее количество лет, которое человек проживёт на -0.0625
\item $S = 11.31$
\item Коэффициент корреляции говорит о наличии или отсутствии лишь \textit{линейной} взаимосвязи. $x_2 = a * x_1 ^ 2$  
\item $\rho \approx -0.8$, $\rho_{exact} = -0.8165$. Отрицательная сильная линейная связь
\item $\varepsilon = 0.5$. При увеличении цены маркера на 1\%, срок его службы вырастет на 0.5\%
\end{enumerate}

\textbf{Вариант 18}
\begin{enumerate}
\item Коэффициент корреляции говорит о наличии или отсутствии лишь \textit{линейной} взаимосвязи. $x ^ 2 + y ^ 2 = r^2$
\item $\rho \approx 0.25$, $\rho_{exact} = 0.2483$. Положительная слабая линейная связь
\item $\varepsilon = 0.6$. При увеличении $x$ на 1\%, $y$ изменится на 0.6\%
\item $S = 44.53$
\item $y = 72.3+-0.06\cdot x$. Каждая дополнительная выкуренная сигарета изменяет среднее количество лет, которое человек проживёт на -0.06
\end{enumerate}

\textbf{Вариант 19}
\begin{enumerate}
\item $S = 14.56$
\item $y = 0.1584 + -0.048\cdot x$. При увеличении ВВП страны на 1 млрд. долларов, ИЧР в среднем вырастёт на -0.048
\item $\varepsilon = 6.2$. При увеличении $x$ на 1\%, $y$ изменится на 6.2\%
\item $\rho \approx -0.8$, $\rho_{exact} = -0.8$. Отрицательная сильная линейная связь
\item Положительная умеренная (средняя) линейная. $\symit{\rho_{exact} = 0.5763}$
\end{enumerate}

\textbf{Вариант 20}
\begin{enumerate}
\item $y = 70.5+-0.125\cdot x$. Каждая дополнительная выкуренная сигарета изменяет среднее количество лет, которое человек проживёт на -0.125
\item $S = 6.86$
\item $\varepsilon = 0.4$. При увеличении цены маркера на 1\%, срок его службы вырастет на 0.4\%
\item $\rho \approx -0.2$, $\rho_{exact} = -0.199$. Отрицательная слабая линейная связь
\item Лаборант был неправ. Коэффициент корреляции говорит о наличии или отсутствии лишь \textit{линейной} взаимосвязи. $y = cos(x)$
\end{enumerate}

\end{document}