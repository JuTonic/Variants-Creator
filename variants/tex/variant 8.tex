\documentclass{article}

\usepackage{geometry}
\geometry{
    a4paper,
    includehead=true,
    headsep=3.5mm,
    top=10mm,
    left=20mm,
    right=20mm,
    bottom=20mm
}

\usepackage[russian]{babel}
\sloppy
    
\usepackage{fancyhdr}
\fancyhf{}
\fancypagestyle{fancy}{
	\fancyhead[L]{\textit{Самостоятельная работа 6}}
	\fancyhead[R]{\textit{Вариант 8}}
	\fancyfoot[L]{\thepage}
	\fancyfoot[R]{\textit{курс "основы статистических наблюдений", 2023}}
	\renewcommand{\headrulewidth}{0.20ex}
}
\pagestyle{fancy}
        
\usepackage{fontspec-xetex}
\setmainfont{Open Sans}
    
\usepackage{booktabs}
    
\usepackage{amsmath}
\usepackage{unicode-math}
\setmathfont[Scale=1.2]{Cambria Math}
    
\usepackage{caption}
\captionsetup[table]{name=Таблица, aboveskip=3pt, labelfont={it}, font={it}, justification=raggedleft, singlelinecheck=off}
\captionsetup[figure]{labelformat=empty, margin=0pt, skip = -12pt}

\setlength\parindent{0pt}

\usepackage{booktabs}
\setlength\heavyrulewidth{0.20ex}
\setlength\cmidrulewidth{0.10ex}
\setlength\lightrulewidth{0.10ex}

\usepackage{tabularx}
\newcolumntype{Y}{>{\centering\arraybackslash}X}
\def\tabularxcolumn#1{m{#1}}

\usepackage{svg}

\usepackage{tcolorbox}

\renewcommand{\baselinestretch}{1.5}

\usepackage{makecell}

\usepackage{underscore}

\usepackage{multirow}

\usepackage{enumitem}
\setlist{nolistsep}

\renewcommand{\theenumi}{\alph{enumi}}

\begin{document}


\mbox{}

\vspace{-36pt}

\begin{center}
	\begin{tcolorbox}[colback=white, boxrule=0.20ex, sharp corners = all, height=25pt, colframe=black, valign=top]
		\begin{center}
			Фамилия Имя:\hspace{1.5pt}\rule{190pt}{0pt}\hspace{50pt}Группа:\hspace{1.5pt}\rule{60pt}{0pt}
		\end{center}
	\end{tcolorbox}
\end{center}
\vspace{3pt}

\textbf{Задача 1 (2 б.).} В таблице \ref{task8} представлены данные о численности населения трёх крупнейших городов некоторого региона. Рассчитайте обобщающий показатель централизации (индекс Герфиндаля-Хиршмана) и проинтерпретируйте результат. Ответ округлите до четырёх знаков после запятой\\

\begin{minipage}{\textwidth}
\captionof{table}{}
\centering
\begin{tabularx}{0.4\textwidth}{YY}
\toprule
\textbf{Город} & \textbf{Численность населения, тыс. чел.} \\
\midrule
А & 293 \\

Б & 1831 \\

В & 2409 \\
\addlinespace
\textit{Всего} & \textit{4533} \\
\bottomrule
\end{tabularx}
\label{task8}
\end{minipage} \\[35pt]

\textbf{Задача 2 (2 б.).} Пользуясь данными из таблицы \ref{task7}, в которой представлено распределение населения по совокупному доходу, рассчитайте и проинтерпретируйте коэффициент Лоренса. Ответ округлите до четырёх знаков после запятой\\

\begin{minipage}{\textwidth}
\captionof{table}{}
\centering
\begin{tabularx}{0.8\textwidth}{YYYYY}
\toprule
\small\textbf{Доля населения, ($\symbfit{d_x}$)} & \small\textbf{Доля в совокупном доходе, ($\symbfit{d_y}$)} & $\symbfit{d_y^H}$ & $\symbfit{d_x\cdot d_y}$ & $\symbfit{d_x\cdot d_y^H}$ \\
\midrule
0.25 & 0.02 & 0.02 & 0.005 & 0.005 \\

0.25 & 0.06 & 0.08 & 0.015 & 0.02 \\

0.25 & 0.07 & 0.15 & 0.0175 & 0.0375 \\

0.25 & 0.85 & 1.0 & 0.2125 & 0.25 \\
\addlinespace
\textit{Всего:} & -- & -- & \textit{0.25} & \textit{0.3125} \\
\bottomrule
\end{tabularx}
\label{task7}
\end{minipage} \\[35pt]

\textbf{Задача 3 (2 б.).} По данным о динамике структуры доходов бюджета (таблица \ref{task1}), рассчитайте:
\begin{enumerate}[leftmargin=40pt]
\item Линейных коэффициент <<абсолютных>> структурных сдвигов за период 2021--2023,
\item Квадратический коэффициент <<абсолютных>> структурных сдвигов за период 2022--2023.\medskip
\end{enumerate}

Ответ округлите до двух знаков после запятой. Сформулируйе выводы.\\

\begin{minipage}{\textwidth}
\captionof{table}{}
\centering
\begin{tabularx}{0.8\textwidth}{rYYY}
\toprule
 & \textbf{2021, \%} & \textbf{2022, \%} & \textbf{2023, \%} \\
\cmidrule(lr){2-2}\cmidrule(lr){3-3}\cmidrule(lr){4-4}
Нефтегазовые доходы & 44.7 & 46.1 & 46.9 \\

Налоги на прибыль и доходы & 48.4 & 45.9 & 49.5 \\

Прочее & 6.9 & 8.0 & 3.6 \\
\bottomrule
\end{tabularx}
\label{task1}
\end{minipage} \\[35pt]

\textbf{Задача 4 (4 б.).} В таблице \ref{task5} представлена структура предпочитаемых населением регионов А и Б видов транспорта. Пользуясь этими данными, рассчитайте и проинтерпретируйте интегральный коэффициент К. Гатева. Ответ округлите до четырёх знаков после запятой.\\

\begin{minipage}{\textwidth}
\captionof{table}{}
\centering
\begin{tabularx}{0.7\textwidth}{rYY}
\toprule
 & \textbf{Регион А, \%} & \textbf{Регион Б, \%} \\
\cmidrule(lr){2-2}\cmidrule(lr){3-3}
Личный автомобиль & 50 & 50 \\

Общественный транспорт & 47 & 47 \\

Другое (такси, не пользуюсь транспортом...) & 3 & 3 \\
\bottomrule
\end{tabularx}
\label{task5}
\end{minipage} \\[35pt]

\end{document}