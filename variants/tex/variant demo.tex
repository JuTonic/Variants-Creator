\documentclass{article}

\usepackage{geometry}
\geometry{
    a4paper,
    includehead=true,
    headsep=3.5mm,
    top=10mm,
    left=20mm,
    right=20mm,
    bottom=20mm
}

\usepackage[russian]{babel}
\sloppy
    
\usepackage{fancyhdr}
\fancyhf{}
\fancypagestyle{fancy}{
	\fancyhead[L]{\textit{Экзамен}}
	\fancyhead[R]{\textit{Демо}}
	\fancyfoot[L]{\thepage}
	\fancyfoot[R]{\textit{курс "основы статистических наблюдений", 2023}}
	\renewcommand{\headrulewidth}{0.20ex}
}
\pagestyle{fancy}
        
\usepackage{fontspec-xetex}
\setmainfont{Open Sans}
    
\usepackage{booktabs}
    
\usepackage{amsmath}
\usepackage{unicode-math}
\setmathfont[Scale=1.2]{Cambria Math}
    
\usepackage{caption}
\captionsetup[table]{name=Таблица, aboveskip=3pt, labelfont={it}, font={it}, justification=raggedleft, singlelinecheck=off}
\captionsetup[figure]{labelformat=empty, margin=0pt, skip = -12pt}

\setlength\parindent{0pt}

\usepackage{booktabs}
\setlength\heavyrulewidth{0.20ex}
\setlength\cmidrulewidth{0.10ex}
\setlength\lightrulewidth{0.10ex}

\usepackage{tabularx}
\newcolumntype{Y}{>{\centering\arraybackslash}X}
\def\tabularxcolumn#1{m{#1}}

\usepackage{svg}

\usepackage{tcolorbox}

\renewcommand{\baselinestretch}{1.5}

\usepackage{makecell}

\usepackage{underscore}

\usepackage{multirow}

\usepackage{enumitem}
\setlist{nolistsep}

\renewcommand{\theenumi}{\alph{enumi}}

\begin{document}


\mbox{}

\vspace{-36pt}

\begin{center}
	\begin{tcolorbox}[colback=white, boxrule=0.20ex, sharp corners = all, height=25pt, colframe=black, valign=top]
		\begin{center}
			Фамилия Имя:\hspace{1.5pt}\rule{190pt}{0pt}\hspace{50pt}Группа:\hspace{1.5pt}\rule{60pt}{0pt}
		\end{center}
	\end{tcolorbox}
\end{center}
\vspace{3pt}

\textbf{Задача 1} Основные категории статистики как наукиОтвет: \\

\textbf{Задача 2} Статистическое исследование как категория статистики. Этапы статистического исследования. Источники данных для статистического исследованияОтвет: \\

\textbf{Задача 3} Статистическое наблюдение как метод статистики. Его формы, виды и способыОтвет: \\

\textbf{Задача 4} Статистическая сводка и группировкаОтвет: \\

\textbf{Задача 5} Ряды распределения: основные понятия, виды, методика построенияОтвет: \\

\textbf{Задача 6} Средняя величина: определение, сущность. Средняя арифметическая: виды, применение, свойстваОтвет: \\

\textbf{Задача 7} Средняя величина: определение, сущность. Средние гармоническая, геометрическая, хронологическая: виды, применениеОтвет: \\

\textbf{Задача 8} Показатели вариации: определение, видыОтвет: \\

\textbf{Задача 9} Анализ формы распределения с помощью средних величин и показателей вариацииОтвет: \\

\textbf{Задача 10} Анализ динамики: понятие и классификация рядов динамики, их показатели.Ответ: \\

\textbf{Задача 11} Изучение взаимосвязи: основные понятия, корреляционный анализОтвет: \\

\textbf{Задача 12} Изучение взаимосвязи: основные понятия, регрессионный анализОтвет: \\

\textbf{Задача 13} Экономические индексы: определение, классификация, индексы структурных сдвигов и пространственно-территориального сопоставленияОтвет: \\

\textbf{Задача 14} Выборочные наблюдения: определение, область применения, основные понятияОтвет: \\

\textbf{Задача 15} Выборочные наблюдения: определение, собственно-случайная выборкаОтвет: \\

\textbf{Задача 16} Табличный способ визуализации данных: суть, структура, оформление, видыОтвет: \\

\textbf{Задача 17} Графический способ визуализации данных: суть, структура, оформление, видыОтвет: \\

\textbf{Задача 18} Структурные средние. МодаОтвет: \\

\textbf{Задача 19} Структурные средние. МедианаОтвет: \\

\textbf{Задача 20} Анализ динамики: компоненты ряда динамики, метод скользящей среднейОтвет: \\

\textbf{Задача 21} Анализ динамики: компоненты ряда динамики, метод наименьших квадратов (на примере прямой)Ответ: \\

\textbf{Задача 22} Анализ динамики: методы экстраполяции данныхОтвет: \\

\textbf{Задача 23} Изучение взаимосвязи между качественными признаками для таблиц сопряженности 2х2Ответ: \\

\textbf{Задача 24} Изучение взаимосвязи между качественными признаками для таблиц сопряженности более, чем 2х2Ответ: \\

\textbf{Задача 25} Изучение взаимосвязи между ранговыми признакамиОтвет: \\

\textbf{Задача 26} Анализ структуры: определение, классификация структур, основные показатели структурных измененийОтвет: \\

\textbf{Задача 27} Анализ структуры: определение, классификация структур, сводные показатели оценки структурных сдвиговОтвет: \\

\textbf{Задача 28} Анализ структуры: определение, классификация структур, показатели концентрации и централизацииОтвет: \\

\textbf{Задача 29} Экономические индексы: определение, классификация, индивидуальные индексыОтвет: \\

\textbf{Задача 30} Экономические индексы: определение, классификация, сводные индексыОтвет: \\

\textbf{Задача 31} По данным таблицы \ref{task31} постройте линейное уравнение регресии индекса человеческого развития (ИЧР) на ВВП страны. Коэффициенты уравнения округлите до четырёх знаков после запятой. Проинтерпретируйте коэффициент при $x$\\

\begin{minipage}{\textwidth}
\captionof{table}{}
\footnotesize
\centering
\begin{tabularx}{0.8\textwidth}{YYY}
\toprule
№ страны & ВВП, млрд \$, $x$ & ИЧР, $y$ \\
\cmidrule(lr){1-1}\cmidrule(lr){2-2}\cmidrule(lr){3-3}
1 & 52 & 0.6 \\

2 & 84 & 0.6 \\

3 & 36 & 0.3 \\

4 & 94 & 0.6 \\

5 & 51 & 0.7 \\

6 & 110 & 0.9 \\

7 & 34 & 0.3 \\

8 & 52 & 0.3 \\

9 & 46 & 0.4 \\

10 & 41 & 0.3 \\
\addlinespace[0.3ex]\bottomrule
\end{tabularx}
\label{task31}
\end{minipage} \\[35pt]

Ответ: $\hat y_i = 0.116+0.0064\cdot x_i$. При увеличении ВВП страны на 1 млрд. долларов, ИЧР в среднем увеличится на 0.0064\\

\textbf{Задача 32} В таблице \ref{task32} представлены данные о среднем количестве сигарет, которое курильщик выкуривал в день ($x$) и возраст, до которого он дожил ($y$). Постройте линейное уравнение регрессии $y$ на $x$. Коэффициенты уравнения округлите до четырёх знаков после запятой. Проинтерпретируйте коэффициент при $x$\\

\begin{minipage}{\textwidth}
\captionof{table}{}
\footnotesize
\centering
\begin{tabularx}{0.8\textwidth}{YYY}
\toprule
№ & \makecell{Число выкуриваемых\\ в день сигарет, шт, $x$} & \makecell{Продолжительность\\ жизни, лет $y$} \\
\cmidrule(lr){1-1}\cmidrule(lr){2-2}\cmidrule(lr){3-3}
1 & 4 & 72 \\

2 & 36 & 68 \\

3 & 21 & 71 \\

4 & 21 & 68 \\

5 & 16 & 68 \\

6 & 20 & 70 \\

7 & 12 & 71 \\

8 & 14 & 70 \\

9 & 39 & 66 \\

10 & 17 & 73 \\
\addlinespace[0.3ex]\bottomrule
\end{tabularx}
\label{task32}
\end{minipage} \\[35pt]

Ответ: $\hat y_i = 72.7-0.15\cdot x_i$. Каждая дополнительная выкуренная сигарета снизится среднюю продолжительность жизни на 0.15 лет\\

\textbf{Задача 33} Постройте регрессию $y_i = a_0 + a_1 \cdot \dfrac{1}{x}$, пользуясь данными таблицы \ref{task33}. Коэффициенты округлите до четырёх знаков после запятой. Проинтерпретируйте коэффициент при $1/x$.\\

\begin{minipage}{\textwidth}
\captionof{table}{}
\footnotesize
\centering
\begin{tabularx}{0.8\textwidth}{YYY}
\toprule
№ & $y$ & $x$ \\
\cmidrule(lr){1-1}\cmidrule(lr){2-2}\cmidrule(lr){3-3}
1 & 246 & 1/66 \\

2 & 237 & 1/61 \\

3 & 218 & 1/56 \\

4 & 354 & 1/102 \\

5 & 347 & 1/99 \\

6 & 224 & 1/59 \\

7 & 350 & 1/99 \\

8 & 259 & 1/70 \\

9 & 265 & 1/72 \\

10 & 252 & 1/66 \\
\addlinespace[0.3ex]\bottomrule
\end{tabularx}
\label{task33}
\end{minipage} \\[35pt]

Ответ: $\hat y_i = 50.2+3.0\cdot \dfrac{1}{x_i}$. При увеличении $\dfrac{1}{x}$ на единицу, среднее значение $y$ увеличится на 3.0\\

\textbf{Задача 34} Рассчитайте среднее отклонение $S$, для наблюдаемых и предсказанных значений из таблицы \ref{task34}. Ответ округлите до двух знаков после запятой\\

\begin{minipage}{\textwidth}
\captionof{table}{Значения $y$ и $\hat y$}
\centering
\begin{tabularx}{0.8\textwidth}{YYYYYYY}
\toprule
 & 1 & 2 & 3 & 4 & 5 & Сумма \\
\cmidrule(lr){1-1}\cmidrule(lr){2-6}\cmidrule(lr){7-7}
$y$ & 84 & 116 & 126 & 135 & 76 & 537 \\

$\hat y$ & 80 & 117 & 120 & 141 & 70 & 528 \\
\bottomrule
\end{tabularx}
\label{task34}
\end{minipage} \\[35pt]

Ответ: $S = 5.0$\\

\textbf{Задача 35} По таблице \ref{task1}, в которой приведены данные о скорости бега спортсмена ($x$) и соответствующие им значения пульса ($y$), рассчитайте коэффициент корреляции. Ответ округлите до четырёх знаков после запятой. Охарактеризуйте направление и силу связи между величинами.\\

\begin{minipage}{\textwidth}
\captionof{table}{Cкорость бега ($x$) и частота пульса ($y$) спортсмена}
\centering
\begin{tabularx}{0.8\textwidth}{lYYYY}
\toprule
Скорость бега, км/ч, $x$ & 9 & 7 & 7 & 9 \\
\addlinespace
Пульс, уд/м, $y$ & 176 & 127 & 147 & 154 \\
\bottomrule
\end{tabularx}
\label{task1}
\end{minipage} \\[35pt]

Ответ: $\rho_{exact} = 0.7997$. сильная положительная связь\\

\textbf{Задача 36} В таблице \ref{task2} приведены данные о водителях грузовиков: их возраст ($x$) и средняя скорость, с которой они ездят ($y$). Рассчитайте коэффициент корреляции между указанными величинами. Ответ округлите до четырёх знаков после запятой. Охарактеризуйте направление и силу свящи между величинами.\\

\begin{minipage}{\textwidth}
\captionof{table}{Возраст ($x$) и средней скорости вождения ($y$) водителей грузовиков}
\centering
\begin{tabularx}{0.8\textwidth}{lYYYY}
\toprule
Возраст ($x$) & 53 & 65 & 65 & 69 \\
\addlinespace
Скорость вождения, км/ч ($y$) & 50 & 55 & 48 & 51 \\
\midrule
Возраст * Скорость вождения ($xy$) & 2650 & 3575 & 3120 & 3519 \\
\bottomrule
\end{tabularx}
\label{task2}
\end{minipage} \\[35pt]

Ответ: $\rho \approx 0.2$, $\rho_{exact} = 0.1961$. слабая положительная связь\\

\textbf{Задача 37} В таблице \ref{task3} представлены данные о широте, на которой располагается город ($x$), и его средней годовой температуре ($y$). Рассчитатайте коэффициент корреляции между указанными величинами. Ответ округлите до двух знаков после запятой. Охарактеризуйте направление и силу связи между величинами.\\

\begin{minipage}{\textwidth}
\captionof{table}{Широта ($x$) и средняя годовая температура ($y$) городов}
\centering
\begin{tabularx}{0.8\textwidth}{lYYYY}
\toprule
№ города $i$ & 1 & 2 & 3 & 4 \\
\cmidrule(lr){1-1}\cmidrule(lr){2-5}
Широта, градусы ($x$) & 54 & 48 & 48 & 54 \\
\addlinespace\bottomrule
\end{tabularx}
\label{task37}
\end{minipage} \\[35pt]

Ответ: $\rho_{exact} = -0.7845$. сильная отрицательная связь\\

\textbf{Задача 38} Четыре студента решили проверить, как количество часов, потраченное на компьютерные игры ($x$), влияет на итоговую оценку ($y$). В течение семестра они замеряли, сколько часов каждый из них проводит за компьютеромыми играми, в итоге получив следующие данные: \begin{itemize} \item первый студент наиграл $36$ часов и получил $4.5 балла$ балла из пяти возможных, \item второй наиграл $44$ часов и получил $4.7 балла$ балла, \item третий наиграл $42$ часов и получил $4.1 балла$ балла, \item четвертый наиграл $50$ часов и получил $2.7 балла$ балла. \end{itemize} Найдите коэффициент корреляции между временем, потраченным на игры, и итоговой оценкой. Охарактеризуйте силу и форму связи между величинами.\\

Ответ: $\rho_{exact} = -0.7682$. сильная отрицательная связь\\

\textbf{Задача 39} По таблице \ref{task39}, в которой представлены данные об испытаниях новой вакцины, рассчитайте коэффициенты ассоциации и контингенции. Ответ округлите до четырёх знаков после запятой. Сформулируйте выводы.\\

\begin{minipage}{\textwidth}
	\centering
	\aboverulesep=0ex
	\belowrulesep=0ex
	\captionof{table}{}
	\begin{tabularx}{0.6\textwidth}{rrYYr}
		& & \multicolumn{2}{c}{\small\makecell{\textbf{Выявлено наличие} \\[-5pt] \textbf{антител}}} & \\
		\cmidrule(l{-0.4pt}){3-4}
		& \multicolumn{1}{c|}{} & Да & Нет & \textit{Итого} \\
		\cmidrule{2-2}
		\multirow{2}*{\textbf{\small Вакцинировался}} & Да & 24 & 72 & \textit{96} \\
	 	& Нет & 21 & 9 & \textit{30} \\
		\addlinespace[1ex]
		& \textit{Итого} & \textit{45} & \textit{81} & \textit{126} \\
	\end{tabularx}
  \label{task39}
\end{minipage} \\[35pt]Ответ: $A = -0.75$, $K = -0.4$. Выявлена связь между фактом вакцинации и наличием у испытуемого антител\\

\textbf{Задача 40} По результатам опроса респондентов из Москвы и Санкт-Петербурга о том, пользуются ли они сервисами онлайн доставки, была составлена таблица \ref{task40}. По представленным данным рассчитайте коэффициенты ассоциации и контингенции и проинтерпретируйте их. Ответ округлите до четырёх знаков после запятой\\

\begin{minipage}{\textwidth}
	\centering
	\aboverulesep=0ex
	\belowrulesep=0ex
	\captionof{table}{}
	\begin{tabularx}{0.7\textwidth}{rrYYr}
		& & \multicolumn{2}{c}{\small\makecell{\textbf{Пользуется} \\[-5pt] \textbf{онлайн доставкой}}} & \\
		\cmidrule(l{-0.4pt}){3-4}
		& \multicolumn{1}{c|}{} & Да & Нет & \textit{Итого} \\
		\cmidrule{2-2}
		\multirow{2}*{\textbf{\small Город проживания}} & Москва & 29 & 6 & \textit{35} \\
	 	& Санкт-Петербург & 19 & 9 & \textit{28} \\
		\addlinespace[1ex]
		& \textit{Итого} & \textit{48} & \textit{15} & \textit{63} \\
	\end{tabularx}
  \label{task40}
\end{minipage} \\[35pt]Ответ: $A = 0.392$, $K = 0.175$. Наличие связи между фактом использованием сервисов онлайн доставок и городом проживания респондента не выявлено\\

\textbf{Задача 41} По результатам проверки 84 мясокомбинатов в разных частях города на соблюдение норм СанПиН\'а, была составлена таблица \ref{task3}. Пользуясь представленными данными, рассчитайте коэффициенты контингенции и ассоциации. Ответ округлите до четырёх знаков после запятой\\

\begin{minipage}{\textwidth}
	\centering
	\aboverulesep=0ex
	\belowrulesep=0ex
	\captionof{table}{}
	\begin{tabularx}{0.6\textwidth}{rrYYr}
		& & \multicolumn{2}{c}{\small\makecell{\textbf{Обнаружены} \\[-5pt] \textbf{нарушения}}} & \\
		\cmidrule(l{-0.4pt}){3-4}
		& \multicolumn{1}{c|}{} & Да & Нет & \textit{Итого} \\
		\cmidrule{2-2}
		\multirow{2}*{\textbf{Расположен на}} & Севере & 16 & 14 & \textit{30} \\
	 	& Юге & 48 & 6 & \textit{54} \\
		\addlinespace[1ex]
		& \textit{Итого} & \textit{64} & \textit{20} & \textit{84} \\
	\end{tabularx}
  \label{task3}
\end{minipage} \\[35pt]Ответ: $A = -0.75$, $K = -0.4$. Выявлена связь между расположением мясокомбината и наличием нарушений норм СанПиН'а\\

\textbf{Задача 42} По результатам опроса сельских и городских жителей о их музыкальных предпочтениях была составлена таблица \ref{task4}. Рассчитайте коэффициенты взаимной сопряжённости Пирсона и Чупрова и проинтерпретируйте полученный результат. Ответ округлите до четырёх знаков после запятой\\

\begin{minipage}{\textwidth}
            \aboverulesep=0ex
            \belowrulesep=0ex
            \captionof{table}{}
            \centering
            \begin{tabularx}{0.9\textwidth}{rcYYYYYr}
                & & \multicolumn{5}{c}{\textbf{Слушают}} & \\
                \cmidrule(l{-0.4pt}){3-7}
                & \multicolumn{1}{c|}{} & Поп & Инди & Рок & Кантри & Классика & \textit{Итого} \\
                \cmidrule{2-2}
                \multirow{2}*{\textbf{Проживают в}} & Городе & 88 & 75 & 28 & 67 & 29 & \textit{287} \\
                & Селе & 92 & 24 & 23 & 9 & 16 & \textit{164} \\
                \addlinespace[1ex]
                & \textit{Итого} & \textit{180} & \textit{99} & \textit{51} & \textit{76} & \textit{45} & \textit{451} \\
            \end{tabularx}
            \label{task4}
        \end{minipage} \\[35pt]Ответ: $K_\text{п} \approx 0.3001$, $K_\text{ч} \approx 0.2225$. Связи между фактом проживания респондента в селе или в городе и его предпочтениями в музыке не выявлено\\

\textbf{Задача 43} По данным случайного опроса прохожих разных возрастных категорий о величине их дохода была составлена таблица \ref{task43}. Рассчитайте коэффициенты взаимной сопряжённости Пирсона и Чупрова. Ответ округлите до четырёх знаков после запятой\\

\begin{minipage}{\textwidth}
            \aboverulesep=0ex
            \belowrulesep=0ex
            \captionof{table}{}
            \centering
            \begin{tabularx}{0.8\textwidth}{rcYYYr}
                & & \multicolumn{3}{c}{\textbf{Доход}} & \\
                \cmidrule(l{-0.4pt}){3-5}
                & \multicolumn{1}{c|}{} & Низкий & Средний & Высокий & \textit{Итого} \\
                \cmidrule{2-2}
                \multirow{3}*{\textbf{Возраст}} & 18 -- 35 & 59 & 46 & 24 & \textit{129} \\
                & 35 -- 65 & 16 & 41 & 80 & \textit{137} \\
                & >65 & 45 & 26 & 98 & \textit{169} \\
                \addlinespace[1ex]
                & \textit{Итого} & \textit{120} & \textit{113} & \textit{202} & \textit{435} \\
            \end{tabularx}
            \label{task43}
        \end{minipage} \\[35pt]Ответ: $K_\text{п} \approx 0.3755$, $K_\text{ч} \approx 0.2865$. Выявлено наличие связи между возрастом респондента и его уровнем дохода\\

\textbf{Задача 44} По таблице \ref{task44}, в которой представлены данные об инвестициях компаний в основной капитал и соответствующие им уровни выпуска, рассчитайте ранговые коэффициенты Кэнделла и Спирмена. Ответ округлите до четырёх знаков после запятой\\

\begin{minipage}{\textwidth}
\captionof{table}{}
\centering
\begin{tabularx}{0.8\textwidth}{lYYYYY}
\toprule
Компания & 1 & 2 & 3 & 4 & 5 \\
\midrule
Инвестиции в основной капитал, руб. & 2.4 & 2.5 & 3.3 & 3.7 & 4.1 \\

Выпуск, шт. & 8.2 & 2.9 & 9.7 & 11.6 & 10.4 \\
\bottomrule
\end{tabularx}
\label{task44}
\end{minipage} \\[35pt]

Ответ: $\tau = 0.6$. $\rho = 0.8$\\

\textbf{Задача 45} В таблице \ref{task45} приведены данные о спросе на разные товары одной категории в зависимости от их цены. Рассчитайте ранговые коэффициенты Спирмена и Кэнделла. Ответ округлите до четырёх знаков после запятой\\

\begin{minipage}{\textwidth}
\captionof{table}{}
\centering
\begin{tabularx}{0.8\textwidth}{lYYYYY}
\toprule
Товар & 1 & 2 & 3 & 4 & 5 \\
\midrule
Цена, руб. & 103 & 129 & 152 & 164 & 181 \\

Спрос, шт. & 2468 & 2954 & 2552 & 2200 & 1128 \\

$R_x$ & 1 & 2 & 3 & 4 & 5 \\

$R_y$ & 3 & 5 & 4 & 2 & 1 \\
\bottomrule
\end{tabularx}
\label{task45}
\end{minipage} \\[35pt]

Ответ: $\tau = -0.6$. $\rho = -0.7$\\

\textbf{Задача 46} В таблице \ref{task46} приведены данные о стоимостях облигаций и соответствующие им процентные ставки. Рассчитайте ранговые коэффициенты Спирмена и Кэнделла. Ответ округлите до четырёх знаков после запятой\\

\begin{minipage}{\textwidth}
\captionof{table}{}
\centering
\begin{tabularx}{0.8\textwidth}{lYYYYY}
\toprule
Товар & 1 & 2 & 3 & 4 & 5 \\
\midrule
Процентная ставка, руб. & 0.125 & 0.143 & 0.167 & 0.5 & 1.0 \\

Стоимость облигации, шт. & 310 & 303 & 318 & 277 & 286 \\

$R_x$ & 1 & 2 & 3 & 4 & 5 \\

$R_y$ & 4 & 3 & 5 & 1 & 2 \\
\bottomrule
\end{tabularx}
\label{task46}
\end{minipage} \\[35pt]

Ответ: $\tau = -0.4$. $\rho = -0.6$\\

\textbf{Задача 47} По данным о динамике структуры доходов бюджета (таблица \ref{task1}), рассчитайте:
\begin{enumerate}[leftmargin=40pt]
\item Квадратический коэффициент <<абсолютных>> структурных сдвигов за период 2017--2021,
\item Линейный коэффициент <<абсолютных>> структурных сдвигов за период 2013--2017,
\item Линейных коэффициент <<абсолютных>> структурных сдвигов за период 2013--2021.\medskip
\end{enumerate}

Ответ округлите до двух знаков после запятой. Сформулируйе выводы.\\

\begin{minipage}{\textwidth}
\captionof{table}{}
\centering
\begin{tabularx}{0.8\textwidth}{rYYY}
\toprule
 & \textbf{2013, \%} & \textbf{2017, \%} & \textbf{2021, \%} \\
\cmidrule(lr){2-2}\cmidrule(lr){3-3}\cmidrule(lr){4-4}
Нефтегазовые доходы & 42.2 & 48.2 & 48.8 \\

Налоги на прибыль и доходы & 49.6 & 42.6 & 42.5 \\

Прочее & 8.2 & 9.2 & 8.7 \\
\bottomrule
\end{tabularx}
\label{task1}
\end{minipage} \\[35pt]

Ответ: \begin{enumerate} \item $\sigma_\text{2017--2021}\approx 0.4546$ п.п. В период 2017--2021 гг. удельный вес отдельных направлений поступления доходов в бюджет изменился в среднем на 0.45 процентных пункта
\item $\bar\Delta_\text{2013--2017}\approx 4.6667$ п.п. В период 2013--2017 гг. удельный вес отдельных направлений поступлений доходов в бюджет изменился в среднем на 4.67 процентных пункта\item $\bar\Delta_\text{2013--2021}\approx 1.1667$ п.п. В рассматриваемый период 2013--2021 гг. среднегодовое изменение по всем направлениям поступлений доходов в бюджет составило 1.17 процентных пункта\\\end{enumerate}

\textbf{Задача 48} По таблице \ref{task47}, в которой отражена динамика структуры предприятий города А по их размеру, рассчитайте:
\begin{enumerate}[leftmargin=40pt]
\item Линейных коэффициент <<абсолютных>> структурных сдвигов за период 2015--2019,
\item Квадратический коэффициент <<относительных>> структурных сдвигов за период 2015--2017,
\item Квадратический коэффициент <<абсолютных>> структурных сдвигов за период 2015--2017.\medskip
\end{enumerate}

Ответ округлите до двух знаков после запятой. Сформулируйе выводы.\\

\begin{minipage}{\textwidth}
\captionof{table}{}
\centering
\begin{tabularx}{0.8\textwidth}{rYYY}
\toprule
 & \textbf{2015, \%} & \textbf{2017, \%} & \textbf{2019, \%} \\
\cmidrule(lr){2-2}\cmidrule(lr){3-3}\cmidrule(lr){4-4}
Крупные предприятия & 48.4 & 49.0 & 49.8 \\

Средние предприятия & 42.8 & 42.3 & 46.0 \\

Малые предприятия & 8.8 & 8.7 & 4.2 \\
\bottomrule
\end{tabularx}
\label{task1}
\end{minipage} \\[35pt]

Ответ: \begin{enumerate} \item $\bar\Delta_\text{2015--2019}\approx 0.7667$ п.п. В рассматриваемый период 2015--2019 гг. среднегодовое изменение удельного веса предприятий всех размеров составил 0.77 процентных пункта
\item $\sigma_\text{2015/2017}\approx 1.2006$ п.п. В относительном выражении в период 2015--2017 гг. удельный вес предприятий всех размеров в среднем изменился на 1.2 процентных пункта\item $\sigma_\text{2015--2017}\approx 0.4546$ п.п. В период 2015--2017 гг. удельный вес предприятий отдельных размеров изменился в среднем на 0.45 процентных пункта\end{enumerate}

\textbf{Задача 49} По таблице \ref{task3}, в которой отражена динамика структуры персонала предприятия, рассчитайте:
\begin{enumerate}[leftmargin=40pt]
\item Квадратический коэффициент <<относительных>> структурных сдвигов за период 2016--2020,
\item Линейных коэффициент <<абсолютных>> структурных сдвигов за период 2012--2020,
\item Квадратический коэффициент <<абсолютных>> структурных сдвигов за период 2012--2016.\medskip
\end{enumerate}

Ответ округлите до двух знаков после запятой. Сформулируйе выводы.\\

\begin{minipage}{\textwidth}
\captionof{table}{}
\centering
\begin{tabularx}{0.8\textwidth}{rYYY}
\toprule
 & \textbf{2012, \%} & \textbf{2016, \%} & \textbf{2020, \%} \\
\cmidrule(lr){2-2}\cmidrule(lr){3-3}\cmidrule(lr){4-4}
Менеджеры & 44.1 & 43.0 & 40.9 \\

Высококвалифицированные кадры & 49.6 & 48.9 & 49.8 \\

Рабочие & 6.3 & 8.1 & 9.3 \\
\bottomrule
\end{tabularx}
\label{task49}
\end{minipage} \\[35pt]

Ответ: \begin{enumerate} \item $\sigma_\text{2016/2020}\approx 5.4489$ п.п. В относительном выражении в период 2016--2020 гг. удельный вес сотрудников всех категорий в среднем изменился на 5.45 процентных пункта
\item $\bar\Delta_\text{2012--2020}\approx 0.2667$ п.п. В рассматриваемый период 2012--2020 гг. среднегодовое изменение удельного веса сотрудников всех категорий составил 0.27 процентных пункта\item $\sigma_\text{2012--2016}\approx 2.5351$ п.п. В период 2012--2016 гг. удельный вес отдельных категорий сотрудников изменился в среднем на 2.54 процентных пункта\end{enumerate}

\textbf{Задача 50} В таблице \ref{task4} представлена структура потребления товаров разных категорий для населений городов А и Б. Пользуясь этими данными, рассчитайте и проинтерпретируйте интегральный коэффициент К. Гатева, индекс Салаи и индекс Рябцева. Ответ округлите до четырёх знаков после запятой.\\

\begin{minipage}{\textwidth}
\captionof{table}{}
\centering
\begin{tabularx}{0.7\textwidth}{rYY}
\toprule
 & \textbf{Город А, \%} & \textbf{Город Б, \%} \\
\cmidrule(lr){2-2}\cmidrule(lr){3-3}
Предметы роскоши & 26 & 17 \\

Нормальные блага & 44 & 34 \\

Товары первой необходимости & 30 & 49 \\
\bottomrule
\end{tabularx}
\label{task4}
\end{minipage} \\[35pt]

Ответ: \begin{enumerate}
\item $J_s\approx 0.2977$. Наблюдаются существенный уровень различий структур потребления в городах А и Б,
\item $K_s\approx 0.3887$. Наблюдается значительный уровень различий струкрут потребления в городах А и Б,
\item $I_r\approx 0.2859$. Наблюдаются существенный уровень различий структур потребления в городах А и Б.
\end{enumerate}

\textbf{Задача 51} В таблице \ref{task5} представлена структура предпочитаемых населением регионов А и Б видов транспорта. Пользуясь этими данными, рассчитайте и проинтерпретируйте интегральный коэффициент К. Гатева, индекс Салаи и индекс Рябцева. Ответ округлите до четырёх знаков после запятой.\\

\begin{minipage}{\textwidth}
\captionof{table}{}
\centering
\begin{tabularx}{0.7\textwidth}{rYY}
\toprule
 & \textbf{Регион А, \%} & \textbf{Регион Б, \%} \\
\cmidrule(lr){2-2}\cmidrule(lr){3-3}
Личный автомобиль & 31 & 27 \\

Общественный транспорт & 34 & 49 \\

Другое (такси, не пользуюсь транспортом...) & 35 & 24 \\
\bottomrule
\end{tabularx}
\label{task4}
\end{minipage} \\[35pt]

Ответ: \begin{enumerate}
\item $J_s\approx 0.2073$. Наблюдаются существенный уровень различий структур предпочитаемых видов транспорта в городах А и Б,
\item $K_s\approx 0.3065$. Наблюдается значительный уровень различий струкрут предпочитаемых видов транспорта в городах А и Б,
\item $I_r\approx 0.222$. Наблюдаются существенный уровень различий структур предпочитаемых видов транспорта в городах А и Б.
\end{enumerate}

\textbf{Задача 52} В таблице \ref{task51} представлено распределение населений стран А и Б по их классовой принадлежности. Пользуясь этими данными, рассчитайте и проинтерпретируйте интегральный коэффициент К. Гатева, индекс Салаи и индекс Рябцева. Ответ округлите до четырёх знаков после запятой.\\

\begin{minipage}{\textwidth}
\captionof{table}{}
\centering
\begin{tabularx}{0.7\textwidth}{rYY}
\toprule
 & \textbf{Регион А, \%} & \textbf{Регион Б, \%} \\
\cmidrule(lr){2-2}\cmidrule(lr){3-3}
Личный автомобиль & 6 & 15 \\

Общественный транспорт & 16 & 44 \\

Другое (такси, не пользуюсь транспортом...) & 78 & 41 \\
\bottomrule
\end{tabularx}
\label{task4}
\end{minipage} \\[35pt]

Ответ: \begin{enumerate}
\item $J_s\approx 0.4734$. Наблюдается значительный уровень различий струкрут классовой принадлежности в городах А и Б,
\item $K_s\approx 0.6194$. Наблюдается весьма значительный уровень различий структур классовой принадлежности в городах А и Б,
\item $I_r\approx 0.4872$. Наблюдается значительный уровень различий струкрут классовой принадлежности в городах А и Б.
\end{enumerate}

\textbf{Задача 53} Пользуясь данными из таблицы \ref{task7}, в которой представлено распределение населения по совокупному доходу, рассчитайте и проинтерпретируйте коэффициенты Лоренца и Джини. Ответ округлите до четырёх знаков после запятой\\

\begin{minipage}{\textwidth}
\captionof{table}{}
\centering
\begin{tabularx}{0.8\textwidth}{YY}
\toprule
\small\textbf{Доля населения, ($\symbfit{d_x}$)} & \small\textbf{Доля в совокупном доходе, ($\symbfit{d_y}$)} \\
\midrule
0.25 & 0.03 \\

0.25 & 0.03 \\

0.25 & 0.04 \\

0.25 & 0.9 \\
\addlinespace\bottomrule
\end{tabularx}
\label{task53}
\end{minipage} \\[35pt]

Ответ: \begin{enumerate} 
 \item $L= 0.65$. Результат указывает на очень высокую концетрацию доходов населения
 \item $G= 0.655$. Результат указывает на очень высокую концетрацию доходов населения
 \end{enumerate}

\end{document}