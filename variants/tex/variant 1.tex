\documentclass{article}

\usepackage{geometry}
\geometry{
    a4paper,
    includehead=true,
    headsep=3.5mm,
    top=10mm,
    left=20mm,
    right=20mm,
    bottom=20mm
}

\usepackage[russian]{babel}
\sloppy
    
\usepackage{fancyhdr}
\fancyhf{}
\fancypagestyle{fancy}{
	\fancyhead[L]{\textit{Самостоятельная работа 6}}
	\fancyhead[R]{\textit{Вариант 1}}
	\fancyfoot[L]{\thepage}
	\fancyfoot[R]{\textit{курс "основы статистических наблюдений", 2023}}
	\renewcommand{\headrulewidth}{0.20ex}
}
\pagestyle{fancy}
        
\usepackage{fontspec-xetex}
\setmainfont{Open Sans}
    
\usepackage{booktabs}
    
\usepackage{amsmath}
\usepackage{unicode-math}
\setmathfont[Scale=1.2]{Cambria Math}
    
\usepackage{caption}
\captionsetup[table]{name=Таблица, aboveskip=3pt, labelfont={it}, font={it}, justification=raggedleft, singlelinecheck=off}
\captionsetup[figure]{labelformat=empty, margin=0pt, skip = -12pt}

\setlength\parindent{0pt}

\usepackage{booktabs}
\setlength\heavyrulewidth{0.20ex}
\setlength\cmidrulewidth{0.10ex}
\setlength\lightrulewidth{0.10ex}

\usepackage{tabularx}
\newcolumntype{Y}{>{\centering\arraybackslash}X}
\def\tabularxcolumn#1{m{#1}}

\usepackage{svg}

\usepackage{tcolorbox}

\renewcommand{\baselinestretch}{1.5}

\usepackage{makecell}

\usepackage{underscore}

\usepackage{multirow}

\usepackage{enumitem}
\setlist{nolistsep}

\renewcommand{\theenumi}{\alph{enumi}}

\begin{document}


\mbox{}

\vspace{-36pt}

\begin{center}
	\begin{tcolorbox}[colback=white, boxrule=0.20ex, sharp corners = all, height=25pt, colframe=black, valign=top]
		\begin{center}
			Фамилия Имя:\hspace{1.5pt}\rule{190pt}{0pt}\hspace{50pt}Группа:\hspace{1.5pt}\rule{60pt}{0pt}
		\end{center}
	\end{tcolorbox}
\end{center}
\vspace{3pt}

\textbf{Задача 1 (2 б.).} Пользуясь данными из таблицы \ref{task7}, в которой представлено распределение населения по совокупному доходу, рассчитайте и проинтерпретируйте индекс Джини. Ответ округлите до четырёх знаков после запятой\\

\begin{minipage}{\textwidth}
\captionof{table}{}
\centering
\begin{tabularx}{0.8\textwidth}{YYYYY}
\toprule
\small\textbf{Доля населения, ($\symbfit{d_x}$)} & \small\textbf{Доля в совокупном доходе, ($\symbfit{d_y}$)} & $\symbfit{d_y^H}$ & $\symbfit{d_x\cdot d_y}$ & $\symbfit{d_x\cdot d_y^H}$ \\
\midrule
0.25 & 0.08 & 0.08 & 0.02 & 0.02 \\

0.25 & 0.08 & 0.16 & 0.02 & 0.04 \\

0.25 & 0.13 & 0.29 & 0.0325 & 0.0725 \\

0.25 & 0.71 & 1.0 & 0.1775 & 0.25 \\
\addlinespace
\textit{Всего:} & -- & -- & \textit{0.25} & \textit{0.3825} \\
\bottomrule
\end{tabularx}
\label{task7}
\end{minipage} \\[35pt]

\textbf{Задача 2 (2 б.).} В таблице \ref{task9} представлены данные о количестве патентов, зарегистрированных в странах А, Б и В. Рассчитайте обобщающий показатель централизации (индекс Герфиндаля-Хиршмана) и проинтерпретируйте результат. Ответ округлите до четырёх знаков после запятой\\

\begin{minipage}{\textwidth}
\captionof{table}{}
\centering
\begin{tabularx}{0.4\textwidth}{YY}
\toprule
\textbf{Страна} & \textbf{Количество патентов, шт.} \\
\midrule
А & 2813 \\

Б & 2554 \\

В & 1008 \\
\addlinespace
\textit{Всего} & \textit{6375} \\
\bottomrule
\end{tabularx}
\label{task9}
\end{minipage} \\[35pt]

\textbf{Задача 3 (2 б.).} По данным о динамике структуры доходов бюджета (таблица \ref{task1}), рассчитайте:
\begin{enumerate}[leftmargin=40pt]
\item Квадратический коэффициент <<абсолютных>> структурных сдвигов за период 2018--2019,
\item Линейный коэффициент <<абсолютных>> структурных сдвигов за период 2019--2020.\medskip
\end{enumerate}

Ответ округлите до двух знаков после запятой. Сформулируйе выводы.\\

\begin{minipage}{\textwidth}
\captionof{table}{}
\centering
\begin{tabularx}{0.8\textwidth}{rYYY}
\toprule
 & \textbf{2018, \%} & \textbf{2019, \%} & \textbf{2020, \%} \\
\cmidrule(lr){2-2}\cmidrule(lr){3-3}\cmidrule(lr){4-4}
Нефтегазовые доходы & 47.5 & 46.7 & 49.8 \\

Налоги на прибыль и доходы & 47.6 & 46.9 & 41.0 \\

Прочее & 4.9 & 6.4 & 9.2 \\
\bottomrule
\end{tabularx}
\label{task1}
\end{minipage} \\[35pt]

\textbf{Задача 4 (4 б.).} В таблице \ref{task4} представлена структура потребления товаров разных категорий для населений городов А и Б. Пользуясь этими данными, рассчитайте и проинтерпретируйте индекс Салаи. Ответ округлите до четырёх знаков после запятой.\\

\begin{minipage}{\textwidth}
\captionof{table}{}
\centering
\begin{tabularx}{0.7\textwidth}{rYY}
\toprule
 & \textbf{Город А, \%} & \textbf{Город Б, \%} \\
\cmidrule(lr){2-2}\cmidrule(lr){3-3}
Предметы роскоши & 9 & 4 \\

Нормальные блага & 16 & 44 \\

Товары первой необходимости & 75 & 52 \\
\bottomrule
\end{tabularx}
\label{task4}
\end{minipage} \\[35pt]

\end{document}