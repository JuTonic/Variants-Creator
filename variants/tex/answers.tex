\documentclass{article}

\usepackage{geometry}
\geometry{
    a4paper,
    includehead=true,
    headsep=3.5mm,
    top=10mm,
    left=20mm,
    right=20mm,
    bottom=20mm
}

\usepackage[russian]{babel}
\sloppy
    
\usepackage{fancyhdr}
\fancyhf{}
\fancypagestyle{fancy}{
	\fancyhead[L]{\textit{Самостоятельная работа 6}}
	\fancyhead[R]{\textit{Ответы}}
	\fancyfoot[L]{\thepage}
	\fancyfoot[R]{\textit{курс "основы статистических наблюдений", 2023}}
	\renewcommand{\headrulewidth}{0.20ex}
}
\pagestyle{fancy}
        
\usepackage{fontspec-xetex}
\setmainfont{Open Sans}
    
\usepackage{booktabs}
    
\usepackage{amsmath}
\usepackage{unicode-math}
\setmathfont[Scale=1.2]{Cambria Math}
    
\usepackage{caption}
\captionsetup[table]{name=Таблица, aboveskip=3pt, labelfont={it}, font={it}, justification=raggedleft, singlelinecheck=off}
\captionsetup[figure]{labelformat=empty, margin=0pt, skip = -12pt}

\setlength\parindent{0pt}

\usepackage{booktabs}
\setlength\heavyrulewidth{0.20ex}
\setlength\cmidrulewidth{0.10ex}
\setlength\lightrulewidth{0.10ex}

\usepackage{tabularx}
\newcolumntype{Y}{>{\centering\arraybackslash}X}
\def\tabularxcolumn#1{m{#1}}

\usepackage{svg}

\usepackage{tcolorbox}

\renewcommand{\baselinestretch}{1.5}

\usepackage{makecell}

\usepackage{underscore}

\usepackage{multirow}

\usepackage{enumitem}
\setlist{nolistsep}

\renewcommand{\theenumi}{\alph{enumi}}

\begin{document}


\mbox{}

\vspace{-36pt}

\begin{center}
	\begin{tcolorbox}[colback=white, boxrule=0.20ex, sharp corners = all, height=25pt, colframe=black, valign=top]
		\begin{center}
			Фамилия Имя:\hspace{1.5pt}\rule{190pt}{0pt}\hspace{50pt}Группа:\hspace{1.5pt}\rule{60pt}{0pt}
		\end{center}
	\end{tcolorbox}
\end{center}
\vspace{3pt}

\twocolumn\textbf{Вариант 1}
\begin{enumerate}
\item $G= 0.485$. Результат указывает на относительно высокую концентрацию доходов населения
\item $J_H \approx 0.3802$. Наблюдается высокая степень централизации количества патентов по странам
\item \begin{enumerate} \item $\sigma_\text{2018--2019}\approx 1.0614$ п.п. В период 2018--2019 гг. удельный вес отдельных направлений поступления доходов в бюджет изменился в среднем на 1.06 процентных пункта
\item $\bar\Delta_\text{2019--2020}\approx 3.9333$ п.п. В период 2018--2019 гг. удельный вес отдельных направлений поступлений доходов в бюджет изменился в среднем на 3.93 процентных пункта\end{enumerate}
\item $J_s\approx 0.6216$. Наблюдается весьма значительный уровень различий структур потребления в городах А и Б.
\end{enumerate}

\textbf{Вариант 2}
\begin{enumerate}
\item $J_H \approx 0.4428$. Наблюдается высокая степень централизации прибыли между филиалами
\item $K_s\approx 0.048$. Наблюдается весьма низкий уровень различий структур предпочитаемых видов транспорта в городах А и Б.
\item $L= 0.48$. Результат указывает на относительно высокую концентрацию доходов населения
\item \begin{enumerate} \item $\sigma_\text{2015/2016}\approx 50.0683$ п.п. В относительном выражении в период 2015--2016 гг. удельный вес предприятий всех размеров в среднем изменился на 50.07 процентных пункта
\item $\bar\Delta_\text{2015--2017}= 1.6$ п.п. В рассматриваемый период 2015--2016 гг. среднегодовое изменение удельного веса предприятий всех размеров составил 1.6 процентных пункта\end{enumerate}
\end{enumerate}

\textbf{Вариант 3}
\begin{enumerate}
\item \begin{enumerate} \item $\bar\Delta_\text{2012--2018}\approx 2.5444$ п.п. В рассматриваемый период 2012--2018 гг. среднегодовое изменение удельного веса сотрудников всех категорий составил 2.54 процентных пункта
\item $\sigma_\text{2012--2015}= 7.12$ п.п. В период 2012--2018 гг. удельный вес отдельных категорий сотрудников изменился в среднем на 7.12 процентных пункта\end{enumerate}
\item $L= 0.31$. Результат указывает на относительно высокую концентрацию доходов населения
\item $I_r\approx 0.6652$. Наблюдается весьма значительный уровень различий структур классовой принадлежности в городах А и Б.
\item $J_H \approx 0.4184$. Наблюдается высокая степень централизации населения
\end{enumerate}

\textbf{Вариант 4}
\begin{enumerate}
\item $L= 0.28$. Результат указывает на относительно высокую концентрацию доходов населения
\item $I_r\approx 0.3765$. Наблюдается значительный уровень различий струкрут потребления в городах А и Б.
\item $J_H \approx 0.4606$. Наблюдается высокая степень централизации населения
\item \begin{enumerate} \item $\sigma_\text{2013/2016}\approx 80.0239$ п.п. В относительном выражении в период 2013--2016 гг. удельный вес предприятий всех размеров в среднем изменился на 80.02 процентных пункта
\item $\bar\Delta_\text{2013--2019}\approx 0.3444$ п.п. В рассматриваемый период 2013--2016 гг. среднегодовое изменение удельного веса предприятий всех размеров составил 0.34 процентных пункта\end{enumerate}
\end{enumerate}

\textbf{Вариант 5}
\begin{enumerate}
\item $J_s\approx 0.8698$. Структуры классовой принадлежности в городах А и Б противоположны.
\item $L= 0.46$. Результат указывает на относительно высокую концентрацию доходов населения
\item $J_H \approx 0.4128$. Наблюдается высокая степень централизации прибыли между филиалами
\item \begin{enumerate} \item $\bar\Delta_\text{2014--2018}= 0.9$ п.п. В рассматриваемый период 2014--2018 гг. среднегодовое изменение по всем направлениям поступлений доходов в бюджет составило 0.9 процентных пункта
\item $\sigma_\text{2014/2016}\approx 4.932$ п.п. В относительном выражении в период 2014--2018 гг. удельный вес каждой статьи поступлений доходов в бюджет в среднем изменился на 4.93 процентных пункта\end{enumerate}
\end{enumerate}

\textbf{Вариант 6}
\begin{enumerate}
\item $J_H \approx 0.4585$. Наблюдается высокая степень централизации количества патентов по странам
\item $K_s\approx 0.0546$. Наблюдается весьма низкий уровень различий структур предпочитаемых видов транспорта в городах А и Б.
\item \begin{enumerate} \item $\bar\Delta_\text{2015--2016}= 2.8$ п.п. В период 2015--2016 гг. удельный вес отдельных категорий сотрудников изменился в среднем на 2.8 процентных пункта
\item $\sigma_\text{2016--2017}= 6.98$ п.п. В период 2015--2016 гг. удельный вес отдельных категорий сотрудников изменился в среднем на 6.98 процентных пункта\end{enumerate}
\item $L= 0.54$. Результат указывает на очень высокую концетрацию доходов населения
\end{enumerate}

\textbf{Вариант 7}
\begin{enumerate}
\item $L= 0.42$. Результат указывает на относительно высокую концентрацию доходов населения
\item \begin{enumerate} \item $\bar\Delta_\text{2017--2019}= 10.4$ п.п. В период 2017--2019 гг. удельный вес предприятий отдельных размеров изменился в среднем на 10.4 процентных пункта
\item $\sigma_\text{2019/2021}\approx 54.6877$ п.п. В относительном выражении в период 2017--2019 гг. удельный вес предприятий всех размеров в среднем изменился на 54.69 процентных пункта\end{enumerate}
\item $I_r\approx 0.8453$. Структуры классовой принадлежности в городах А и Б противоположны.
\item $J_H \approx 0.3553$. Наблюдается высокая степень централизации количества патентов по странам
\end{enumerate}

\textbf{Вариант 8}
\begin{enumerate}
\item $J_H \approx 0.4498$. Наблюдается высокая степень централизации населения
\item $L= 0.6$. Результат указывает на очень высокую концетрацию доходов населения
\item \begin{enumerate} \item $\bar\Delta_\text{2021--2023}= 1.1$ п.п. В рассматриваемый период 2021--2023 гг. среднегодовое изменение по всем направлениям поступлений доходов в бюджет составило 1.1 процентных пункта
\item $\sigma_\text{2022--2023}\approx 3.3146$ п.п. В период 2021--2023 гг. удельный вес отдельных направлений поступления доходов в бюджет изменился в среднем на 3.31 процентных пункта\end{enumerate}
\item $K_s\approx 0.0437$. Наблюдается весьма низкий уровень различий структур предпочитаемых видов транспорта в городах А и Б.
\end{enumerate}

\textbf{Вариант 9}
\begin{enumerate}
\item $G= 0.485$. Результат указывает на относительно высокую концентрацию доходов населения
\item $I_r\approx 0.6797$. Наблюдается весьма значительный уровень различий структур потребления в городах А и Б.
\item \begin{enumerate} \item $\bar\Delta_\text{2018--2022}= 3.55$ п.п. В рассматриваемый период 2018--2022 гг. среднегодовое изменение удельного веса сотрудников всех категорий составил 3.55 процентных пункта
\item $\sigma_\text{2020--2022}= 0.78$ п.п. В период 2018--2022 гг. удельный вес отдельных категорий сотрудников изменился в среднем на 0.78 процентных пункта\end{enumerate}
\item $J_H \approx 0.4651$. Наблюдается высокая степень централизации прибыли между филиалами
\end{enumerate}

\textbf{Вариант 10}
\begin{enumerate}
\item \begin{enumerate} \item $\sigma_\text{2019/2020}\approx 23.647$ п.п. В относительном выражении в период 2019--2020 гг. удельный вес каждой статьи поступлений доходов в бюджет в среднем изменился на 23.65 процентных пункта
\item $\bar\Delta_\text{2020--2021}= 4.4$ п.п. В период 2019--2020 гг. удельный вес отдельных направлений поступлений доходов в бюджет изменился в среднем на 4.4 процентных пункта\end{enumerate}
\item $J_s\approx 0.4096$. Наблюдается значительный уровень различий струкрут классовой принадлежности в городах А и Б.
\item $J_H \approx 0.5154$. Наблюдается очень высокая централизация количества патентов по странам
\item $G= 0.405$. Результат указывает на относительно высокую концентрацию доходов населения
\end{enumerate}

\textbf{Вариант 11}
\begin{enumerate}
\item $I_r\approx 0.0074$. Структуры предпочитаемых видов транспорта в городах А и Б тождественны.
\item \begin{enumerate} \item $\bar\Delta_\text{2012--2020}\approx 0.8833$ п.п. В рассматриваемый период 2012--2020 гг. среднегодовое изменение удельного веса сотрудников всех категорий составил 0.88 процентных пункта
\item $\sigma_\text{2016/2020}= 108.41$ п.п. В относительном выражении в период 2012--2020 гг. удельный вес сотрудников всех категорий в среднем изменился на 108.41 процентных пункта\end{enumerate}
\item $G= 0.545$. Результат указывает на очень высокую концетрацию доходов населения
\item $J_H \approx 0.3683$. Наблюдается высокая степень централизации населения
\end{enumerate}

\textbf{Вариант 12}
\begin{enumerate}
\item $J_H \approx 0.3503$. Наблюдается высокая степень централизации прибыли между филиалами
\item $J_s\approx 0.1833$. Наблюдаются существенный уровень различий структур потребления в городах А и Б.
\item \begin{enumerate} \item $\bar\Delta_\text{2018--2020}= 7.7$ п.п. В рассматриваемый период 2018--2020 гг. среднегодовое изменение удельного веса предприятий всех размеров составил 7.7 процентных пункта
\item $\sigma_\text{2018/2019}\approx 39.9785$ п.п. В относительном выражении в период 2018--2020 гг. удельный вес предприятий всех размеров в среднем изменился на 39.98 процентных пункта\end{enumerate}
\item $G= 0.59$. Результат указывает на очень высокую концетрацию доходов населения
\end{enumerate}

\textbf{Вариант 13}
\begin{enumerate}
\item \begin{enumerate} \item $\sigma_\text{2022--2023}= 26.86$ п.п. В период 2022--2023 гг. удельный вес отдельных категорий сотрудников изменился в среднем на 26.86 процентных пункта
\item $\bar\Delta_\text{2021--2022}\approx 7.8667$ п.п. В период 2022--2023 гг. удельный вес отдельных категорий сотрудников изменился в среднем на 7.87 процентных пункта\end{enumerate}
\item $L= 0.29$. Результат указывает на относительно высокую концентрацию доходов населения
\item $J_H \approx 0.4125$. Наблюдается высокая степень централизации прибыли между филиалами
\item $K_s\approx 0.0491$. Наблюдается весьма низкий уровень различий структур предпочитаемых видов транспорта в городах А и Б.
\end{enumerate}

\textbf{Вариант 14}
\begin{enumerate}
\item $G= 0.525$. Результат указывает на очень высокую концетрацию доходов населения
\item \begin{enumerate} \item $\bar\Delta_\text{2018--2019}= 27.2$ п.п. В период 2018--2019 гг. удельный вес предприятий отдельных размеров изменился в среднем на 27.2 процентных пункта
\item $\sigma_\text{2019/2020}\approx 64.338$ п.п. В относительном выражении в период 2018--2019 гг. удельный вес предприятий всех размеров в среднем изменился на 64.34 процентных пункта\end{enumerate}
\item $J_H \approx 0.4177$. Наблюдается высокая степень централизации количества патентов по странам
\item $J_s\approx 0.6011$. Наблюдается весьма значительный уровень различий структур потребления в городах А и Б.
\end{enumerate}

\textbf{Вариант 15}
\begin{enumerate}
\item $G= 0.505$. Результат указывает на очень высокую концетрацию доходов населения
\item $K_s\approx 0.5719$. Наблюдается весьма значительный уровень различий структур классовой принадлежности в городах А и Б.
\item \begin{enumerate} \item $\bar\Delta_\text{2020--2022}= 1.0$ п.п. В рассматриваемый период 2020--2022 гг. среднегодовое изменение по всем направлениям поступлений доходов в бюджет составило 1.0 процентных пункта
\item $\sigma_\text{2020--2021}\approx 4.3043$ п.п. В период 2020--2022 гг. удельный вес отдельных направлений поступления доходов в бюджет изменился в среднем на 4.3 процентных пункта\end{enumerate}
\item $J_H \approx 0.3794$. Наблюдается высокая степень централизации населения
\end{enumerate}

\textbf{Вариант 16}
\begin{enumerate}
\item $G= 0.61$. Результат указывает на очень высокую концетрацию доходов населения
\item $J_H \approx 0.4809$. Наблюдается высокая степень централизации населения
\item $I_r\approx 0.4788$. Наблюдается значительный уровень различий струкрут потребления в городах А и Б.
\item \begin{enumerate} \item $\bar\Delta_\text{2016--2022}\approx 0.3333$ п.п. В рассматриваемый период 2016--2022 гг. среднегодовое изменение удельного веса сотрудников всех категорий составил 0.33 процентных пункта
\item $\sigma_\text{2019/2022}= 50.33$ п.п. В относительном выражении в период 2016--2022 гг. удельный вес сотрудников всех категорий в среднем изменился на 50.33 процентных пункта\end{enumerate}
\end{enumerate}

\textbf{Вариант 17}
\begin{enumerate}
\item $J_H \approx 0.3496$. Наблюдается высокая степень централизации количества патентов по странам
\item $L= 0.34$. Результат указывает на относительно высокую концентрацию доходов населения
\item $J_s\approx 0.0409$. Наблюдается весьма низкий уровень различий структур предпочитаемых видов транспорта в городах А и Б.
\item \begin{enumerate} \item $\sigma_\text{2018--2019}\approx 3.398$ п.п. В период 2018--2019 гг. удельный вес предприятий отдельных размеров изменился в среднем на 3.4 процентных пункта
\item $\bar\Delta_\text{2018--2020}\approx 11.2667$ п.п. В рассматриваемый период 2018--2019 гг. среднегодовое изменение удельного веса предприятий всех размеров составил 11.27 процентных пункта\end{enumerate}
\end{enumerate}

\textbf{Вариант 18}
\begin{enumerate}
\item \begin{enumerate} \item $\bar\Delta_\text{2014--2022}\approx 0.6167$ п.п. В рассматриваемый период 2014--2022 гг. среднегодовое изменение по всем направлениям поступлений доходов в бюджет составило 0.62 процентных пункта
\item $\sigma_\text{2014/2018}\approx 14.5258$ п.п. В относительном выражении в период 2014--2022 гг. удельный вес каждой статьи поступлений доходов в бюджет в среднем изменился на 14.53 процентных пункта\end{enumerate}
\item $I_r\approx 0.2248$. Наблюдаются существенный уровень различий структур потребления в городах А и Б.
\item $J_H \approx 0.4039$. Наблюдается высокая степень централизации прибыли между филиалами
\item $L= 0.54$. Результат указывает на очень высокую концетрацию доходов населения
\end{enumerate}

\textbf{Вариант 19}
\begin{enumerate}
\item $I_r\approx 0.4403$. Наблюдается значительный уровень различий струкрут классовой принадлежности в городах А и Б.
\item \begin{enumerate} \item $\bar\Delta_\text{2014--2016}= 10.8$ п.п. В период 2014--2016 гг. удельный вес предприятий отдельных размеров изменился в среднем на 10.8 процентных пункта
\item $\sigma_\text{2016--2018}\approx 1.995$ п.п. В период 2014--2016 гг. удельный вес предприятий отдельных размеров изменился в среднем на 1.99 процентных пункта\end{enumerate}
\item $J_H \approx 0.3438$. Наблюдается высокая степень централизации населения
\item $L= 0.5$. Результат указывает на относительно высокую концентрацию доходов населения
\end{enumerate}

\textbf{Вариант 20}
\begin{enumerate}
\item $J_H \approx 0.5998$. Наблюдается очень высокая централизация количества патентов по странам
\item $G= 0.33$. Результат указывает на относительно высокую концентрацию доходов населения
\item \begin{enumerate} \item $\sigma_\text{2015--2018}= 12.96$ п.п. В период 2015--2018 гг. удельный вес отдельных категорий сотрудников изменился в среднем на 12.96 процентных пункта
\item $\bar\Delta_\text{2018--2021}\approx 19.3333$ п.п. В период 2015--2018 гг. удельный вес отдельных категорий сотрудников изменился в среднем на 19.33 процентных пункта\end{enumerate}
\item $K_s\approx 0.8624$. Структуры классовой принадлежности в городах А и Б противоположны.
\end{enumerate}

\textbf{Вариант 21}
\begin{enumerate}
\item \begin{enumerate} \item $\sigma_\text{2012/2016}\approx 12.2245$ п.п. В относительном выражении в период 2012--2016 гг. удельный вес каждой статьи поступлений доходов в бюджет в среднем изменился на 12.22 процентных пункта
\item $\bar\Delta_\text{2016--2020}\approx 2.1333$ п.п. В период 2012--2016 гг. удельный вес отдельных направлений поступлений доходов в бюджет изменился в среднем на 2.13 процентных пункта\end{enumerate}
\item $L= 0.54$. Результат указывает на очень высокую концетрацию доходов населения
\item $J_H \approx 0.3655$. Наблюдается высокая степень централизации прибыли между филиалами
\item $I_r\approx 0.0652$. Наблюдается весьма низкий уровень различий структур предпочитаемых видов транспорта в городах А и Б.
\end{enumerate}

\textbf{Вариант 22}
\begin{enumerate}
\item $J_s\approx 0.0307$. Наблюдается весьма низкий уровень различий структур предпочитаемых видов транспорта в городах А и Б.
\item $L= 0.54$. Результат указывает на очень высокую концетрацию доходов населения
\item $J_H \approx 0.4372$. Наблюдается высокая степень централизации населения
\item \begin{enumerate} \item $\sigma_\text{2010--2014}\approx 23.2174$ п.п. В период 2010--2014 гг. удельный вес предприятий отдельных размеров изменился в среднем на 23.22 процентных пункта
\item $\bar\Delta_\text{2010--2018}= 2.55$ п.п. В рассматриваемый период 2010--2014 гг. среднегодовое изменение удельного веса предприятий всех размеров составил 2.55 процентных пункта\end{enumerate}
\end{enumerate}

\textbf{Вариант 23}
\begin{enumerate}
\item $L= 0.57$. Результат указывает на очень высокую концетрацию доходов населения
\item \begin{enumerate} \item $\sigma_\text{2013--2016}= 11.54$ п.п. В период 2013--2016 гг. удельный вес отдельных категорий сотрудников изменился в среднем на 11.54 процентных пункта
\item $\bar\Delta_\text{2013--2019}\approx 2.6111$ п.п. В рассматриваемый период 2013--2016 гг. среднегодовое изменение удельного веса сотрудников всех категорий составил 2.61 процентных пункта\end{enumerate}
\item $J_H \approx 0.5017$. Наблюдается очень высокая централизация прибыли между филиалами
\item $I_r\approx 0.4562$. Наблюдается значительный уровень различий струкрут классовой принадлежности в городах А и Б.
\end{enumerate}

\textbf{Вариант 24}
\begin{enumerate}
\item \begin{enumerate} \item $\bar\Delta_\text{2021--2023}\approx 3.2667$ п.п. В период 2021--2023 гг. удельный вес отдельных направлений поступлений доходов в бюджет изменился в среднем на 3.27 процентных пункта
\item $\sigma_\text{2019--2021}\approx 0.8602$ п.п. В период 2021--2023 гг. удельный вес отдельных направлений поступления доходов в бюджет изменился в среднем на 0.86 процентных пункта\end{enumerate}
\item $J_s\approx 0.0166$. Структуры предпочитаемых видов транспорта в городах А и Б тождественны.
\item $J_H \approx 0.3504$. Наблюдается высокая степень централизации количества патентов по странам
\item $G= 0.565$. Результат указывает на очень высокую концетрацию доходов населения
\end{enumerate}

\textbf{Вариант 25}
\begin{enumerate}
\item $I_r\approx 0.4868$. Наблюдается значительный уровень различий струкрут потребления в городах А и Б.
\item $G= 0.41$. Результат указывает на относительно высокую концентрацию доходов населения
\item \begin{enumerate} \item $\sigma_\text{2021/2022}\approx 15.9296$ п.п. В относительном выражении в период 2021--2022 гг. удельный вес каждой статьи поступлений доходов в бюджет в среднем изменился на 15.93 процентных пункта
\item $\bar\Delta_\text{2020--2022}\approx 2.4333$ п.п. В рассматриваемый период 2021--2022 гг. среднегодовое изменение по всем направлениям поступлений доходов в бюджет составило 2.43 процентных пункта\end{enumerate}
\item $J_H \approx 0.35$. Наблюдается высокая степень централизации населения
\end{enumerate}

\textbf{Вариант 26}
\begin{enumerate}
\item $G= 0.49$. Результат указывает на относительно высокую концентрацию доходов населения
\item $J_H \approx 0.419$. Наблюдается высокая степень централизации прибыли между филиалами
\item \begin{enumerate} \item $\sigma_\text{2016/2017}\approx 105.1172$ п.п. В относительном выражении в период 2016--2017 гг. удельный вес предприятий всех размеров в среднем изменился на 105.12 процентных пункта
\item $\bar\Delta_\text{2015--2017}\approx 9.1333$ п.п. В рассматриваемый период 2016--2017 гг. среднегодовое изменение удельного веса предприятий всех размеров составил 9.13 процентных пункта\end{enumerate}
\item $I_r\approx 0.731$. Структуры классовой принадлежности в городах А и Б противоположны.
\end{enumerate}

\textbf{Вариант 27}
\begin{enumerate}
\item \begin{enumerate} \item $\sigma_\text{2014/2017}= 55.8$ п.п. В относительном выражении в период 2014--2017 гг. удельный вес сотрудников всех категорий в среднем изменился на 55.8 процентных пункта
\item $\bar\Delta_\text{2011--2017}\approx 2.3111$ п.п. В рассматриваемый период 2014--2017 гг. среднегодовое изменение удельного веса сотрудников всех категорий составил 2.31 процентных пункта\end{enumerate}
\item $J_H \approx 0.4921$. Наблюдается высокая степень централизации количества патентов по странам
\item $I_r\approx 0.2709$. Наблюдаются существенный уровень различий структур потребления в городах А и Б.
\item $L= 0.42$. Результат указывает на относительно высокую концентрацию доходов населения
\end{enumerate}

\end{document}