\documentclass{article}

\usepackage{geometry}
\geometry{
    a4paper,
    includehead=true,
    headsep=3.5mm,
    top=10mm,
    left=20mm,
    right=20mm,
    bottom=20mm
}

\usepackage[russian]{babel}
\sloppy
    
\usepackage{fancyhdr}
\fancyhf{}
\fancypagestyle{fancy}{
	\fancyhead[L]{\textit{Самостоятельная работа 4}}
	\fancyhead[R]{\textit{Ответы}}
	\fancyfoot[L]{\thepage}
	\fancyfoot[R]{\textit{курс "основы статистических наблюдений", 2023}}
	\renewcommand{\headrulewidth}{0.20ex}
}
\pagestyle{fancy}
        
\usepackage{fontspec-xetex}
\setmainfont{Open Sans}
    
\usepackage{booktabs}
    
\usepackage{amsmath}
\usepackage{unicode-math}
\setmathfont[Scale=1.2]{Cambria Math}
    
\usepackage{caption}
\captionsetup[table]{name=Таблица, aboveskip=3pt, labelfont={it}, font={it}, justification=raggedleft, singlelinecheck=off}
\captionsetup[figure]{labelformat=empty, margin=0pt, skip = -12pt}

\setlength\parindent{0pt}

\usepackage{booktabs}
\setlength\heavyrulewidth{0.20ex}
\setlength\cmidrulewidth{0.10ex}
\setlength\lightrulewidth{0.10ex}

\usepackage{tabularx}
\newcolumntype{Y}{>{\centering\arraybackslash}X}
\def\tabularxcolumn#1{m{#1}}

\usepackage{svg}

\usepackage{tcolorbox}

\renewcommand{\baselinestretch}{1.5}

\usepackage{makecell}

\usepackage{multirow}

\usepackage{underscore}

\begin{document}

\vspace*{\fill}

\begin{center}
	\large \textbf{Замечания}
\end{center}

В ответах на задачу на расчёт коэффициента корреляции приведены два значения:
\begin{itemize}
	\item $\rho$ - рассчитан по данным в условии промежуточным результатам. Они округлы до десятых, поэтому итоговый ответ содержит некоторую погрешность
	\item $\rho_{exact}$ - точное значение коэффициента корреляции, которое рассчитано по истинным, неокруглённым значениям.
\end{itemize}
От студента требуется привести только одно из них\\

В ответах на задачу, где нужно по графику определить силу связи между переменными, точное значение коэффициента ($\rho_{exact}$) дано \textbf{для проверяющих}. От студентов его указывать не требуется

\vspace*{\fill}

\newpage

\textbf{Вариант 1}
\begin{enumerate}
\item $\hat y_i = 47.45+3.05\cdot \dfrac{1}{x_i}$. При увеличении $\dfrac{1}{x}$ на единицу, среднее значение $y$ вырастет на 3.05
\item Неправы. ''Из некореллированности не следует независимость'' и/или ''переменные, очевидно, связаны нелинейным уравнением''
\item $S = 3.0$
\item $\rho \approx 0.9$, $\rho_{exact} = 0.8955$. сильная положительная связь
\item $\textup{Э} = 1.9$. При увеличении срока службы маркера на 1\%, его средняя цена вырастет на 1.9\%
\end{enumerate}

\textbf{Вариант 2}
\begin{enumerate}
\item $S = 6.0$
\item сильная положительная связь. $\symit{\rho_{exact} = 0.9541}$
\item $\hat y_i = 57.4-1.5\cdot x_i$. При увеличении максимальной частоты процессора на 1 ГГц, средняя скорость выполнения программы уменьшает на 1.5 секунд
\item $\rho \approx 0.32$, $\rho_{exact} = 0.3138$. умеренная (средняя) положительная линейная
\item $\textup{Э} = 3.3$. При увеличении $x$ на 1\%, среднее значение $y$ вырастет на 3.3\%
\end{enumerate}

\textbf{Вариант 3}
\begin{enumerate}
\item $\rho \approx -0.25$, $\rho_{exact} = -0.2469$. слабая отрицательная связь
\item Неправ. ''Из некореллированности не следует независимость'' и/или ''переменные, очевидно, связаны нелинейным уравнением''
\item $\hat y_i = 73.4-0.16\cdot x_i$. Каждая дополнительная выкуренная сигарета уменьшает среднюю продолжительность жизни на 0.16 лет
\item $S = 9.0$
\item $\textup{Э} = 1.5$. При увеличении срока службы маркера на 1\%, его средняя цена вырастет на 1.5\%
\end{enumerate}

\textbf{Вариант 4}
\begin{enumerate}
\item $\rho \approx -0.8$, $\rho_{exact} = -0.8433$. сильная отрицательная связь
\item $\hat y_i = 0.05+0.01\cdot x_i$. При увеличении ВВП страны на 1 млрд. долларов, ИЧР в среднем увеличится на 0.01
\item $S = 4.0$
\item $\textup{Э} = 1.5$. При увеличении $x$ на 1\%, среднее значение $y$ вырастет на 1.5\%
\item Неправ. ''Из некореллированности не следует независимость'' и/или ''переменные, очевидно, связаны нелинейным уравнением''
\end{enumerate}

\newpage

\textbf{Вариант 5}
\begin{enumerate}
\item $\textup{Э} = 1.0$. При увеличении срока службы маркера на 1\%, его средняя цена вырастет на 1.0\%
\item $S = 5.0$
\item $\rho \approx 0.1$, $\rho_{exact} = 0.0995$. величины некоррелированны
\item $\hat y_i = -0.0256+0.0072\cdot x_i$. При увеличении ВВП страны на 1 млрд. долларов, ИЧР в среднем увеличится на 0.0072
\item сильная отрицательная связь. $\symit{\rho_{exact} = -0.9781}$
\end{enumerate}

\textbf{Вариант 6}
\begin{enumerate}
\item $S = 7.0$
\item Неправ. ''Из некореллированности не следует независимость'' и/или ''переменные, очевидно, связаны нелинейным уравнением''
\item $\textup{Э} = 1.8$. При увеличении $x$ на 1\%, среднее значение $y$ вырастет на 1.8\%
\item $\rho \approx -0.8$, $\rho_{exact} = -0.8433$. сильная отрицательная связь
\item $\hat y_i = 49.1+3.03\cdot \dfrac{1}{x_i}$. При увеличении $\dfrac{1}{x}$ на единицу, среднее значение $y$ вырастет на 3.03
\end{enumerate}

\textbf{Вариант 7}
\begin{enumerate}
\item Неправ. ''Из некореллированности не следует независимость'' и/или ''переменные, очевидно, связаны нелинейным уравнением''
\item $S = 9.0$
\item $\textup{Э} = 2.6$. При увеличении срока службы маркера на 1\%, его средняя цена вырастет на 2.6\%
\item $\hat y_i = 56.6-1.0\cdot x_i$. При увеличении максимальной частоты процессора на 1 ГГц, средняя скорость выполнения программы уменьшает на 1.0 секунд
\item $\rho \approx 0.6$, $\rho_{exact} = 0.5987$. умеренная (средняя) положительная линейная
\end{enumerate}

\textbf{Вариант 8}
\begin{enumerate}
\item $\hat y_i = 70.335-0.105\cdot x_i$. Каждая дополнительная выкуренная сигарета уменьшает среднюю продолжительность жизни на 0.105 лет
\item Неправы. ''Из некореллированности не следует независимость'' и/или ''переменные, очевидно, связаны нелинейным уравнением''
\item $\rho \approx -0.8$, $\rho_{exact} = -0.7845$. сильная отрицательная связь
\item $\textup{Э} = 2.0$. При увеличении $x$ на 1\%, среднее значение $y$ вырастет на 2.0\%
\item $S = 5.0$
\end{enumerate}

\newpage

\textbf{Вариант 9}
\begin{enumerate}
\item $\hat y_i = 59.2-1.98\cdot x_i$. При увеличении максимальной частоты процессора на 1 ГГц, средняя скорость выполнения программы уменьшает на 1.98 секунд
\item сильная отрицательная связь. $\symit{\rho_{exact} = -0.9858}$
\item $S = 4.0$
\item $\rho \approx -0.4$, $\rho_{exact} = -0.3885$. умеренная (средняя) отрицательная связь
\item $\textup{Э} = 0.3$. При увеличении срока службы маркера на 1\%, его средняя цена вырастет на 0.3\%
\end{enumerate}

\textbf{Вариант 10}
\begin{enumerate}
\item $\hat y_i = 69.9-0.06\cdot x_i$. Каждая дополнительная выкуренная сигарета уменьшает среднюю продолжительность жизни на 0.06 лет
\item $\textup{Э} = 2.1$. При увеличении $x$ на 1\%, среднее значение $y$ вырастет на 2.1\%
\item $\rho \approx 0.7$, $\rho_{exact} = 0.6965$. умеренная (средняя) положительная линейная
\item Неправ. ''Из некореллированности не следует независимость'' и/или ''переменные, очевидно, связаны нелинейным уравнением''
\item $S = 7.0$
\end{enumerate}

\textbf{Вариант 11}
\begin{enumerate}
\item Неправы. ''Из некореллированности не следует независимость'' и/или ''переменные, очевидно, связаны нелинейным уравнением''
\item $\textup{Э} = 2.1$. При увеличении срока службы маркера на 1\%, его средняя цена вырастет на 2.1\%
\item $\rho \approx -0.8$, $\rho_{exact} = -0.7845$. сильная отрицательная связь
\item $S = 7.0$
\item $\hat y_i = 0.0776+0.0064\cdot x_i$. При увеличении ВВП страны на 1 млрд. долларов, ИЧР в среднем увеличится на 0.0064
\end{enumerate}

\textbf{Вариант 12}
\begin{enumerate}
\item Неправы. ''Из некореллированности не следует независимость'' и/или ''переменные, очевидно, связаны нелинейным уравнением''
\item $S = 5.0$
\item $\hat y_i = 47.38+3.04\cdot \dfrac{1}{x_i}$. При увеличении $\dfrac{1}{x}$ на единицу, среднее значение $y$ вырастет на 3.04
\item $\textup{Э} = 4.5$. При увеличении $x$ на 1\%, среднее значение $y$ вырастет на 4.5\%
\item $\rho \approx -0.25$, $\rho_{exact} = -0.2504$. слабая отрицательная связь
\end{enumerate}

\newpage

\textbf{Вариант 13}
\begin{enumerate}
\item $S = 7.0$
\item $\textup{Э} = 1.2$. При увеличении срока службы маркера на 1\%, его средняя цена вырастет на 1.2\%
\item $\rho \approx -0.5$, $\rho_{exact} = -0.465$. умеренная (средняя) отрицательная связь
\item Неправ. ''Из некореллированности не следует независимость'' и/или ''переменные, очевидно, связаны нелинейным уравнением''
\item $\hat y_i = 0.15+0.005\cdot x_i$. При увеличении ВВП страны на 1 млрд. долларов, ИЧР в среднем увеличится на 0.005
\end{enumerate}

\textbf{Вариант 14}
\begin{enumerate}
\item Неправ. ''Из некореллированности не следует независимость'' и/или ''переменные, очевидно, связаны нелинейным уравнением''
\item $\hat y_i = 46.8+1.0\cdot x_i$. При увеличении максимальной частоты процессора на 1 ГГц, средняя скорость выполнения программы увеличивает на 1.0 секунд
\item $\rho \approx -0.8$, $\rho_{exact} = -0.8$. сильная отрицательная связь
\item $\textup{Э} = 1.5$. При увеличении $x$ на 1\%, среднее значение $y$ вырастет на 1.5\%
\item $S = 9.0$
\end{enumerate}

\textbf{Вариант 15}
\begin{enumerate}
\item сильная отрицательная связь. $\symit{\rho_{exact} = -0.9951}$
\item $\rho \approx 0.85$, $\rho_{exact} = 0.8479$. сильная положительная связь
\item $\hat y_i = 52.0+3.0\cdot \dfrac{1}{x_i}$. При увеличении $\dfrac{1}{x}$ на единицу, среднее значение $y$ вырастет на 3.0
\item $S = 9.0$
\item $\textup{Э} = 1.2$. При увеличении $x$ на 1\%, среднее значение $y$ вырастет на 1.2\%
\end{enumerate}

\textbf{Вариант 16}
\begin{enumerate}
\item $\rho \approx -0.25$, $\rho_{exact} = -0.249$. слабая отрицательная связь
\item Неправ. ''Из некореллированности не следует независимость'' и/или ''переменные, очевидно, связаны нелинейным уравнением''
\item $\hat y_i = 44.8+3.06\cdot \dfrac{1}{x_i}$. При увеличении $\dfrac{1}{x}$ на единицу, среднее значение $y$ вырастет на 3.06
\item $\textup{Э} = 0.7$. При увеличении срока службы маркера на 1\%, его средняя цена вырастет на 0.7\%
\item $S = 7.0$
\end{enumerate}

\newpage

\textbf{Вариант 17}
\begin{enumerate}
\item $\textup{Э} = 2.1$. При увеличении срока службы маркера на 1\%, его средняя цена вырастет на 2.1\%
\item $\hat y_i = 69.575-0.075\cdot x_i$. Каждая дополнительная выкуренная сигарета уменьшает среднюю продолжительность жизни на 0.075 лет
\item $\rho \approx -0.4$, $\rho_{exact} = -0.3651$. умеренная (средняя) отрицательная связь
\item Неправы. ''Из некореллированности не следует независимость'' и/или ''переменные, очевидно, связаны нелинейным уравнением''
\item $S = 4.0$
\end{enumerate}

\textbf{Вариант 18}
\begin{enumerate}
\item $S = 7.0$
\item $\rho \approx 0.25$, $\rho_{exact} = 0.2469$. слабая положительная связь
\item $\hat y_i = 70.25-0.09\cdot x_i$. Каждая дополнительная выкуренная сигарета уменьшает среднюю продолжительность жизни на 0.09 лет
\item $\textup{Э} = 2.6$. При увеличении $x$ на 1\%, среднее значение $y$ вырастет на 2.6\%
\item Неправ. ''Из некореллированности не следует независимость'' и/или ''переменные, очевидно, связаны нелинейным уравнением''
\end{enumerate}

\textbf{Вариант 19}
\begin{enumerate}
\item $\textup{Э} = 1.7$. При увеличении $x$ на 1\%, среднее значение $y$ вырастет на 1.7\%
\item сильная положительная связь. $\symit{\rho_{exact} = 0.9209}$
\item $\hat y_i = 0.22+0.005\cdot x_i$. При увеличении ВВП страны на 1 млрд. долларов, ИЧР в среднем увеличится на 0.005
\item $S = 7.0$
\item $\rho \approx -0.8$, $\rho_{exact} = -0.8$. сильная отрицательная связь
\end{enumerate}

\textbf{Вариант 20}
\begin{enumerate}
\item $\rho \approx 0.25$, $\rho_{exact} = 0.2495$. слабая положительная связь
\item Неправ. ''Из некореллированности не следует независимость'' и/или ''переменные, очевидно, связаны нелинейным уравнением''
\item $\textup{Э} = 2.4$. При увеличении срока службы маркера на 1\%, его средняя цена вырастет на 2.4\%
\item $\hat y_i = 66.66+0.01\cdot x_i$. Каждая дополнительная выкуренная сигарета увеличивает среднюю продолжительность жизни на 0.01 лет
\item $S = 7.0$
\end{enumerate}

\newpage
\begin{minipage}{\textwidth}
	\centering
	\aboverulesep=0ex
	\belowrulesep=0ex
	\captionof{table}{}
	\begin{tabularx}{0.6\textwidth}{rrYYr}
		& & \multicolumn{2}{c}{\textbf{Выздоровел}} & \\
		\cmidrule(l{-0.3pt}){3-4}
		& \multicolumn{1}{c|}{} & Да & Нет & \textit{Итого} \\
		\cmidrule{2-2}
		\multirow{2}*{\textbf{Вакцинировался}} & Да & 18 & 54 & \textit{72} \\
	 	& Нет & 54 & 54 & \textit{108} \\
		\addlinespace[1ex]
		& \textit{Итого} & \textit{72} & \textit{108} & \textit{180} \\
	\end{tabularx}
\end{minipage} \\[35pt]

\begin{minipage}{\textwidth}
	\aboverulesep=0ex
	\belowrulesep=0ex
	\captionof{table}{}
	\centering
	\begin{tabularx}{0.8\textwidth}{rcYYYYr}
		& & \multicolumn{4}{c}{Y} & \\
		\cmidrule(l{-0.4pt}){3-6}
		& \multicolumn{1}{c|}{} & [`Y1`] & [`Y2`] & [`Y3`] & [`Y4`] & Итого \\
		\cmidrule{2-2}
		\multirow{2}*{X} & X1 & [`n11`] & [`n12`] & [`n13`] & [`n13`] & 301 \\
		& X2 & 62 & 67 & 69 & 24 & 222 \\
		& Итого & 146 & 151 & 163 & 63 & 523 \\
	\end{tabularx}
	\label{None}
\end{minipage} \\[35pt]

\begin{minipage}{\textwidth}
	\centering
	\aboverulesep=0ex
	\belowrulesep=0ex
	\captionof{table}{}
	\begin{tabularx}{0.6\textwidth}{rrYYr}
		& & \multicolumn{2}{c}{\small\makecell{\textbf{Выявлены антитела}}} & \\
		\cmidrule(l{-0.4pt}){3-4}
		& \multicolumn{1}{c|}{} & Да & Нет & \textit{\small Итого} \\
		\cmidrule{2-2}
		\multirow{2}*{\textbf{\small Вакцинировался}} & Да & 3 & 29 & \textit{32} \\
		& Нет & 3 & 21 & \textit{24} \\
		\addlinespace[1ex]
		& \textit{\small Итого} & \textit{6} & \textit{50} & \textit{56} \\
	\end{tabularx}
	\label{task1}
\end{minipage} \\

\newpage

По данным случайного опроса прохожих разных возрастных категорий о величине их дохода была составлена таблица \ref{task6}. Рассчитайте коэффициенты взаимной сопряжённости Пирсона и Чупрова, пользуясь тем, что $\sum_{i, j = 1}\dfrac{n_{ij}}{n_{i*}n_{*j}} \approx \dfrac{9}{8}$. Ответ дайте либо в виде обыкновенных несократимых дробей и/или в виде десятичных дробей, округлённых до четырёх знаков после запятой\\

\begin{minipage}{\textwidth}
	\aboverulesep=0ex
	\belowrulesep=0ex
	\captionof{table}{}
	\centering
	\begin{tabularx}{0.8\textwidth}{rcYYYr}
		& & \multicolumn{3}{c}{\textbf{Доход}} & \\
		\cmidrule(l{-0.4pt}){3-5}
		& \multicolumn{1}{c|}{} & Низкий & Средний & Высокий & \textit{Итого} \\
		\cmidrule{2-2}
		\multirow{3}*{\textbf{Возраст}} & 18 -- 35 & 36 & 67 & 86 & \textit{189} \\
		& 35 -- 65 & 51 & 63 & 14 & \textit{128} \\
		& >65 & 39 & 40 & 83 & \textit{162} \\
		\addlinespace[1ex]
		& \textit{Итого} & \textit{126} & \textit{170} & \textit{183} & \textit{479} \\
	\end{tabularx}
	\label{task6}
\end{minipage} \\[35pt] $K_\text{п} = \dfrac{1}{\sqrt{9}} \approx 0.3333$, $K_\text{ч} = \dfrac{1}{4} = 0.25$

\end{document}